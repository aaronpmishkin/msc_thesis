%% The following is a directive for TeXShop to indicate the main file
%!TEX root = ../main.tex

\chapter{Introduction}\label{ch:Introduction}

Stochastic first-order methods are the most popular optimization algorithms in modern machine learning.
In particular, \ac{SGD}~\citep{robbins1951sgd} and its adaptive variants~\citep{duchi2011adagrad, tieleman2012rmsprop, zeiler2012adadelta, kingma2015adam} are widely used in large-scale supervised learning, where they are frequently referred to as fundamental ``workhorse'' algorithms~\citep{qian2019improvedrates, assran2019sgpush, grosse2015scaling}. 
The main advantage of these methods for machine learning is that they only use the gradient of a single or small sub-sample of training examples to update the model parameters at each iteration.
The computational cost of \ac{SGD} (and variants) is thus independent of the training set size and scales to very large datasets and models.
This property is also why stochastic first-order methods are the dominant approach to training highly expressive models, such as deep neural networks~\citep{zhang2017understanding, bengio2012practical} and non-parametric kernels~\citep{liang2018just, belkin2019datainterp}.

Despite their successes, stochastic first-order methods suffer from two well known problems. 
The step-size, momentum term, and other algorithmic hyper-parameters must be carefully tuned to obtain good performance~\citep{bengio2012practical, schaul2013no, li2019convergence, choi2019empirical}; and
they converge slowly compared to deterministic~\citep{nesterov2004lectures} or variance-reduced algorithms~\citep{leroux2012sag, johnson2013svrg, defazio2014saga} even when well-tuned.
The issue of hyper-parameter tuning for \ac{SGD} is the focus of intense research, with approaches ranging from adaptive step-sizes inspired by online learning~\citep{luo2019adabound, li2019convergence, orabona2017coin} to meta-learning~\citep{baydin2018hypergradient, schraudolph1999local, sutton1992gain, almeida1998parameter, plagianakos2001learning, shao2000rates} and heuristics for online estimation~\citep{schaul2013no, rolinek2018l4, tan2016bb}.
In contrast, the slow convergence of first-order methods in the general stochastic setting cannot be improved, with tight lower-bounds showing \( \Theta(\epsilon^{-4}) \) complexity for convergence to a stationary point~\citep{drori2019complexity, arjevani2019lower}.
This compares poorly to deterministic methods, which are \( \Theta(\epsilon^{-2}) \) for the same problem class~\citep{carmon2019lower}.

Increasing experimental and theoretical evidence shows that the slow optimization speed of stochastic first-order methods is mitigated when training over-parameterized models~\citep{ma2018power, arora2018overparameterization, zhou2019analysis}.
For example, variance-reduced algorithms typically underperform \ac{SGD} in this setting despite using increased memory or computation~\citep{defazio2019effectiveness, ma2018power}. 
A key property of over-parameterized problems is that they satisfy \emph{interpolation}, meaning the model can exactly fit all of the available training data~\citep{belkin2019datainterp}.
While this may seem strong, interpolation has been observed in practice for popular methods such as boosting~\citep{schapire1997boosting}, kernel learning~\citep{belkin2019datainterp}, and over-parameterized neural networks~\citep{belkin2019reconciling, zhang2013gradient}. 
Under additional assumptions, interpolation is a sufficient condition for the strong~\citep{solodov1998incremental, tseng1998incremental, schmidt2013fast} or weak~\citep{vaswani2019fast, bassily2018exponential} growth conditions, which imply the second-moment of the stochastic gradients is bounded by a linear function of either the gradient-norm, or the optimality gap, respectively.
A form of automatic reduction in gradient noise occurs~\citep{liu2020accelerating} when strong or weak growth is satisfied, explaining why variance reduction may be unnecessary. 

The connection between over-parameterization and fast stochastic optimization has led to a wave of interest in analyzing first-order methods under interpolation and weak/strong growth.
A number of works have shown that \ac{SGD} obtains the fast convergence rate of deterministic gradient methods for interpolating models~\citep{schmidt2013fast, bassily2018exponential, vaswani2019fast, cevher2018linear, jain2018accelerating}, while related research has established accelerated rates under an additional requirement for convexity~\citep{liu2020accelerating, vaswani2019fast, jain2018accelerating}.
Sub-sampled second-order methods have also been explored and proved to have local linear convergence for specific function classes~\citep{meng2020fastandfurious}.
These positive results show the interpolation setting is restrictive enough to break the \( \Omega\rbr{\epsilon^{-4}} \) barrier for stochastic first-order methods and attain the optimal \( \Theta(\epsilon^{-1}) \) complexity (up to problem-specific constants) for finding stationary points of smooth, convex functions~\citep{nemirovsky1985optimal, arjevani2016iteration}. 

Despite these rapid advances, theoretical rates and practical performance for \ac{SGD} under interpolation still rely on carefully selected hyper-parameters.
A number of approaches from the general stochastic setting are rapidly being adopted to tackle this problem.
For instance, several works have considered a stochastic version of the Polyak step-size, which does not require knowledge of the smoothness or convexity parameters~\citep{loizou2020sps, berrada2019training}. 
Stochastic line-searches using the classic Armijo condition~\citep{armijo1966ls} have also been proposed and shown to obtain fast convergence under interpolation~\citep{vaswani2019painless}.
Very recently, adaptive variants of \ac{SGD} without momentum have analyzed both with and without the Armijo line-search~\citep{vaswani2020adaptive}.

This thesis analyzes a small, but core group of methods in stochastic optimization under interpolation.
The focus is on first-order algorithms; we consider constant step-size \ac{SGD}, \ac{SGD} with a stochastic Armijo line-search, and a version of Nesterov's accelerated gradient descent with stochastic gradients~\citep{nesterov2004lectures}. 
In nearly all cases, we show that existing analyses can be tightened to yield faster convergence rates with a larger range of step-sizes. 
In the case of acceleration, the improvement is comparable to taking the square-root of the condition number and addresses criticisms that previous analyses yield inferior rates to those of \ac{SGD} in some circumstances~\citep{liu2020accelerating}. 

The thesis is organized as follows: 
In \autoref{ch:interpolation-gc}, we formalize interpolation and the strong/weak growth conditions in the context of general stochastic oracles. 
Connections between interpolation, smoothness properties of the stochastic oracle, and growth of the stochastic gradients are derived.
\autoref{ch:sgd} analyzes the complexity of \ac{SGD} with a fixed step-size, drawing comparisons with previous work as well as convergence rates in the deterministic setting. 
Asymptotic convergence of SGD with a constant step-size to
\begin{inparaenum}[(i)]
\item stationary points of general non-convex functions, and
\item minimizers of convex functions 
\end{inparaenum}
is shown under the strong and weak growth conditions.
Chapters~\ref{ch:line-search} and~\ref{ch:acceleration} then address the convergence of \ac{SGD} with a stochastic Armijo line-search and stochastic accelerated gradient descent, respectively. 
Finally, \autoref{ch:beyond-interpolation} leaves the basic interpolation setting behind and considers structural minimization with interpolating functions as components. 
Convergence to an explicit neighborhood is derived for \( L_2 \)-regularized problems as a special case. 


\section{Preliminaries and Assumptions}\label{sec:setup}

%% Basic Assumptions
This work considers the problem of minimizing a continuous function \( f : \R^d \into \R \) under interpolation conditions.
We assume that \( f \) is bounded below with at least one finite minimizer.
That is, there exists at least one \( \wopt \in \R^d \) such that 
\begin{align*}
    f(w) \geq f(\wopt) \quad \quad \forall w \in \R^d. 
\end{align*}
For functions with with multiple minimizers, we write \( \calX^* = \argmin_{w \in \R^d} f(\w) \) and denote the projection of a point onto the optimal set as \( \Pi_{\calX^*}(\w) \).
Additionally, \( f \) is required to be differentiable and \( L \)-smooth, meaning the mapping \( \w \mapsto \grad(\w) \) is an \( L \)-Lipschitz function,
\[ \norm{\grad(\w) - \grad(v)} \leq L \norm{\w - v} \quad \quad \forall \, \w, v \in \R^d. \]
This is equivalent to the existence of the following quadratic upper-bound on \( f \):
\begin{align*}
    f(v) &\leq f(w) + \abr{\grad(\w), v - \w} + \frac{L}{2}\norm{v - \w}^2 &\forall \, \w,v \in \R^d. \tag{\( L \)-Smoothness} 
    \intertext{At times, we will further assume that \( f \) is convex or \( \mu \)-strongly-convex,}
    f(v) &\geq f(\w) + \abr{\grad(\w), v - \w} &\forall \, \w,v \in \R^d, \tag{Convexity} \\
    f(v) &\geq f(\w) + \abr{\grad(\w), v - \w} + \frac{\mu}{2}\norm{v - \w}^2 &\forall \, \w,v \in \R^d. \tag{\( \mu \)-Strong-Convexity}
\end{align*}
Convexity holds for many problems in machine learning, including linear and logistic regression.

Convexity can be relaxed to a more limited property called invexity.
Formally, we say a differentiable function \( f \) is invex if all stationary points are also global minimizers of \( f \)~\citep{ben1986invexity}, meaning
\[ \grad(\w) = 0 \implies f(\w) \leq f(v) \quad \quad \forall \, v \in \R^d.  \]
Invexity is formally weaker than convexity and follows immediately from the first-order conditions for convexity given above.
The analogue of strong-convexity for invex functions is the Polyak-Łojasiewicz (PL) condition~\citep{karimi2016linear}.
A function is said to be \( \mu \)-PL if
\[ \frac{1}{2 \mu} \norm{\grad(\w)}^2 \leq f(\w) - f(\wopt), \]
holds for all \( \w \in \R^d \).
Again, the Polyak-Łojasiewicz condition is weaker than strong convexity; a \( \mu \)-strongly-convex function is \( \mu \)-PL, but the converse does not hold.

In several cases, it will be useful to interpret our results in the context of finite-sum functions.
The function \( f \) is said to be a finite-sum if it can be written as 
\[ f(\w) = \frac{1}{n} \sum_{i=0}^n f_i(\w),  \]
where the individual (or sub-) functions \( f_i : \R^d \rightarrow \R \) may be strongly-convex, convex, or non-convex depending on the problem.
Such objective functions arise naturally in empirical risk minimization, where \( f_i \) typically corresponds to a single training example \( (x_i, y_i) \) in a training set.
For example, the classic least-squares regression objective can be written as 
\[ f(\w) = \frac{1}{n} \sum_{i=0}^n \rbr{\abr{\w, x_i} - y_i}^2, \] 
which is finite-sum with sub-functions \( f_i(\w) =  \rbr{\abr{\w, x_i} - y_i}^2 \).


\section{Related Work}~\label{sec:related-work}

\noindent \textbf{Interpolation}:
Existing work focuses on interpolation in the context of finite-sum objectives. 
In this setting, \citet{bassily2018exponential} define interpolation in terms of sequences converging to global minima of \( f \). 
They say interpolation holds if, for every sequence \( \seq{\wk} \) satisfying \( \lim_{k \rightarrow \infty} f(\wk) = f(\wopt) \), we also have 
\[ \forall i \in [n], \quad \lim_{k \rightarrow \infty} f_i(\wk) = f(\wopt). \]
\citet{berrada2019training} propose \( \epsilon \)-interpolation, which requires the sub-functions to be lower-bounded and \( \epsilon \) sub-optimal for all \( \wopt \in \argmin f \): 
\[ \forall i \in [n], \quad \inf_\w f_i(\w) \geq B \quad \text{and} \quad f_i(\wopt) - B \leq \epsilon. \]
A larger body of work considers interpolation to hold when the optimal points \( \wopt \) are also stationary points or global minimizers of each sub-function \( f_i \)~\citep{vaswani2019fast, vaswani2019painless, vaswani2020adaptive, meng2020fastandfurious, loizou2020sps},
\begin{gather*}
    \wopt \in \argmin_{\w} f(\w) \implies \wopt \in \argmin_{\w} f_i(\w) \quad \forall i \in [n],\\
          \text{or} \\
    \wopt \in \argmin_{\w} f(\w) \implies \grad_i(\wopt) = 0 \quad \forall i \in [n].
\end{gather*}
We will focus on interpolation conditions of this form, which we extend to general stochastic optimization problems.\\ 

\noindent \textbf{Growth Conditions}:
The strong growth condition was first proposed in the context of incremental gradient methods by \citet{solodov1998incremental} and \citet{tseng1998incremental} as a bound on the squared-norm of the stochastic gradients.
Suppose \( f \) is finite-sum and \( P_i \) is an arbitrary distribution used to sub-sample the finite sum. 
    Then strong growth with parameter \( \rho > 0 \) may be written as 
\[ \text{Maximal Strong Growth:} \quad \norm{\grad_i(\w)}^2 \leq \rho \norm{\grad(\w)}^2, \hspace{8.5em} \]
where the inequality holds almost-surely with respect to \( P_i \).
This definition was later used by~\citet{schmidt2013fast} to derive linear convergence rates for \ac{SGD} with a constant step-size. 
\citet{vaswani2019fast} propose a modified version of strong growth which holds in expectation,
\[ \text{Strong Growth:} \quad \E_{P_i} \sbr{\norm{\grad_i(\w)}^2} \leq \rho \norm{\grad(\w)}^2. \hspace{6.6em} \]
We call this latter condition strong growth and refer to the earlier definition as maximal strong growth.
\citet{vaswani2019fast} also propose the following weak growth condition as a natural relaxation of strong growth: 
\[ \text{Weak Growth}: \quad \E_{P_i}\sbr{\norm{\grad_i(\w)}^2} \leq 2 \rho L \rbr{f(\w) - f(\wopt)}. \hspace{2.2em} \]
\citet{cevher2018linear} refer to strong growth simply as ``the growth condition'' and suggest strong growth with an additive noise parameter as an alternative relaxation, 
\[ \text{Strong Growth + Noise:} \quad \E_{P_i} \sbr{\norm{\grad_i(\w)}^2} \leq \rho \norm{\grad(\w)}^2 + \sigma^2, \hspace{8em} \]
which they also call weak growth.
For clarity, we refer to this assumption as strong growth with additive noise.
For unbiased \( P_i \), strong with noise is slightly weaker than assuming a bound on the variance of the stochastic gradients~\citep{khaled2020better, ghadimi2012optimal1}.
\citet{cevher2018linear} study the convergence of proximal-gradient methods under strong growth with additive noise and also prove that strong growth is both \emph{sufficient and necessary} for \ac{SGD} to converge linearly.\\


\noindent \textbf{Stochastic Acceleration}: 
Many authors have considered accelerating stochastic gradient methods.
\citet{schmidt2011convergence} provide sufficient conditions on gradient noise for acceleration of proximal-gradient methods.
D'Aspremont~(\citeyear{aspremont2008approximate}) and \citet{devolder2014first} investigate accelerated gradient methods under the assumption of approximate oracles with deterministic, bounded errors and derive rates for convergence and error accumulation in this setting.
\citet{honorio2012biased} consider accelerated proximal-gradient methods under biased stochastic oracles and show accumulated errors prevent a high-probability convergence guarantee. 
In contrast, \citet{cohen2018acceleration} assume access to an unbiased oracle with additive gradient noise and propose a noise-resistant acceleration scheme.
Very recently, \citet{chen2020understanding} analyze accelerated methods for strongly-convex functions under strong growth with additive noise. 

An alternative approach to stochastic acceleration splits the convergence rate into stochastic and deterministic parts.
Using this framework, multiple authors have shown that acceleration schemes speed-up convergence for the deterministic component and achieve nearly optimal rates in the stochastic approximation setting~\citep{ghadimi2012optimal1, ghadimi2013optimal2, hu2009accelerated}.
For finite-sum functions, variance reduction techniques can be combined with acceleration to improve convergence on the stochastic component~\citep{allen-zhu2017katyusha, defazio2016pointSaga, allen-zhou2018katyushax, tang2018restkatyusha, shang2018asvrg, kovalev2020loopless}.
Such methods are as fast as deterministic acceleration up to a dependence on \( n \), the number of finite-sum functions.
\edit{Accelerated primal-dual methods have been proposed in this setting under an additional requirement for each \( f_i \) to be Lipschitz-smooth~\citep{zhang2015SPDC, lan2018incremental}.}
Alternative approaches also leveraging finite-sum structure are based on the proximal-point algorithm and include \edit{accelerated SDCA~\citep{shalev-shwarz2014ASDCA}}, Catalyst~\citep{lin2015catalyst}, and accelerated APPA~\citep{frostig2015unregularizing}.

Several works have recently considered acceleration under interpolation.
The most similar to the investigation here is that of \citet{vaswani2019fast}, who analyze a slightly altered version of Nesterov's accelerated gradient descent under the strong growth condition. 
\citet{liu2020accelerating} propose a different modification, called MaSS, and analyze its convergence for convex quadratics as well as strongly-convex functions with additional assumptions. 
These assumptions imply strong growth, but yield hard-to-compare rates. 
\citet{jain2018accelerating} prove accelerated rates for squared-losses under interpolation using a tail-averaging scheme. 
Finally, \citet{assran2020convergence} study the stochastic approximation setting and prove that accelerated gradient descent may fail to accelerate even when interpolation is satisfied.\\

\noindent \textbf{Line Search}:
Line-search is a classic technique to set the step-size in deterministic optimization (see \citet{nocedal1999numerical}), but extensions to stochastic optimization are non-obvious. 
\citet{mahsereci2017pls} use a Gaussian process model to define probabilistic Wolfe conditions~\citep{wolfe1969convergence, wolfe1971convergence}; however, the convergence of \ac{SGD} with this procedure is not known.
\citet{fridovich2019choosing} propose line-search conditions based on golden-section search~\citep{avriel1968golden}, but again only provide empirical evidence for \ac{SGD}. 
\citet{paquette2020stochastic}, \citet{krejic2013line}, and \citet{ogaltsov2019adaptive} prove convergence of \ac{SGD} with the Armijo condition in the general stochastic setting with several caveats. 
\citet{paquette2020stochastic} require adaptive batch-sizes, while \citet{krejic2013line} assume the stochastic gradients are bounded and \citet{ogaltsov2019adaptive} use explicit knowledge of an upper-bound on the variance. 
Other authors consider similar approaches based on multiple function and/or gradient evaluations at each iteration~\citep{friedlander2012hybrid, byrd2012sample, de2016big}.
Such approaches also use growing batch-sizes to ensure convergence.
The work in the thesis follows \citet{vaswani2019painless}, who investigated stochastic versions of the Armijo line-search under interpolation; we provide tighter analyses in a more general setting.\\ 


\noindent \textbf{Asymptotic Convergence}:
The original paper by \citet{robbins1951sgd} establishes asymptotic, almost-sure convergence for \ac{SGD} to the zero of a convex function if the stochastic gradients are bounded and a decreasing step-size is used.
Highly general analyses have since shown almost-sure convergence for non-convex functions when strong growth with additive noise is satisfied~\citep{bertsekas2000gradient, bottou1991approche}.
Alternative work directly derives these conditions from properties of strongly-convex and non-convex functions, also for the purpose of proving almost-sure convergence~\citep{nguyen2018sgd, lei2019stochastic}.
Recently, asymptotic convergence was shown with Adagrad-style step-sizes instead of a fixed, decreasing schedule~\citep{li2019convergence}.


\endinput
