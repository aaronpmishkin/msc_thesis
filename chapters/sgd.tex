%!TEX root = ../main.tex

\chapter{Stochastic Gradient Descent}~\label{ch:sgd}
\todo{All the proofs in this section require individual convexity?}

The discussion in the previous chapter formalized interpolation for general stochastic optimization problems and derived connections between interpolation and the weak/strong growth conditions.  
Now, we turn to the main interest of this work: the complexity of iterative algorithms for \( \rbr{f, \oracle{}} \) when interpolation is satisfied. 
This chapter analyzes the convergence of SGD for strongly-convex and convex functions, while the next two chapters tackle SGD with the Armijo line-search (\autoref{ch:line-search}) and stochastic Nesterov acceleration (\autoref{ch:acceleration}).
In particular, this chapter establishes the following non-asymptotic results for SGD with a fixed step-size: 
\begin{enumerate}
    \item Linear convergence \emph{in-expectation} for strongly-convex \( f \) when \( \rbr{f, \oracle{}} \) satisfies strong growth; this rate is tight with the best-known deterministic rates when \( \rho = 1 \).
    \item \emph{Almost sure} linear convergence for strongly-convex \( f \) and individually-smooth and individually-strongly-convex SFO \oracle{}. 
    \item Sub-linear convergence for convex \( f \) when \( \rbr{f, \oracle{}} \) satisfies weak growth; our proof is simpler than existing analyses and permits a larger step-size.
    \item Faster sub-linear convergence for convex \( f \) when \( \rbr{f, \oracle{}} \) satisfies weak growth \emph{and} \oracle{} is individually-smooth; this rate is tight with the deterministic case when \( \rho = 1 \).
\end{enumerate}
\autoref{sec:almost-sure} at the end of this chapter leaves the finite-time regime and considers asymptotic, almost-sure convergence of SGD with a fixed step-size under strong and weak growth, respectively.
The following properties are proved: 
\begin{enumerate}
    \item SGD converges to a stationary point of \( f \) when \( \rbr{f, \oracle{}} \) satisfies strong growth.
    \item SGD converges to a optimal point \( \wopt \) for convex \( f \) when \( \rbr{f, \oracle{}} \) satisfies weak growth.
\end{enumerate}
This last result is particularly interesting because it concerns convergence of the last input generated by SGD;
we shall see that such results are not straightforward in the non-asymptotic regime, where convergence is instead shown for an averaged input. 

\begin{figure}[t]
\begin{procedure}{Stochastic Gradient Descent}
\item Choose an initial point \( \w_0 \in \R^d \).
\item For each iteration \( k \geq 0 \):
    \begin{enumerate}
        \item Query \oracle{} for \( \grad(\wk, \zk) \).
        \item Update input as\vspace{-1ex}%
            \[ \wkk = \wk - \eta \grad(\wk, \zk). \]
    \end{enumerate}
\end{procedure}
\caption{Stochastic gradient descent with a fixed step-size \( \eta \). Note that only one query to the stochastic oracle is needed per-iteration.}~\label{procedure:sgd}
\end{figure}

Now, let us briefly introduce SGD with a fixed step-size before diving into the analysis.
The basic procedure is given in \autoref{procedure:sgd}; the key components of the algorithm are %
\begin{inparaenum}[i)] 
    \item the use of a stochastic gradient \( \grad(\wk, \zk) \) queried from the oracle at every iteration, and
    \item the fixed step-size \( \eta > 0 \),
    \item the sequence of inputs \( \seq{\wk} \) generated by the algorithm, which are called the \emph{iterates}. 
\end{inparaenum}
The non-asymptotic rates in this chapter will be derived by analyzing the sequence of distances to a minimizer, \( \seq{\norm{\wk - \wopt}^2} \), which we call ``going through iterates''. \autoref{ch:line-search} discusses this proof style in greater detail and with specific reference to the case where \( \eta \) is itself a random variable that depends on \( \grad(\wk, \zk) \). We will see that this introduces significant challenges compared to SGD with a fixed (deterministic) step-size.

\section{Convergence for Strongly-Convex Functions}~\label{sec:sgd-sc}

We first establish the convergence rate of SGD for strongly-convex \( f \) when \( \rbr{f, \oracle} \) satisfies the strong growth condition. 
Recall that, by \autoref{lemma:interpolation_to_sgc}, this is a strictly more general than assuming \oracle{} is \( \Lmax \) individually-smooth and \( \rbr{f, \oracle} \) satisfies minimizer interpolation. 
Specifically, the strong growth condition implies that \( \rbr{f, \oracle{}} \) satisfies stationary-point interpolation (which is equivalent to mixed interpolation and weaker than minimizer interpolation when \( f \) is convex) and it's parameter satisfies \( \rho \leq \frac{\mu}{\Lmax} \) if individual-smoothness and minimizer interpolation hold. 

\begin{restatable}{theorem}{sgcConvex}~\label{thm:sgc-convex}
    Let \( f \) be a \( \mu \)-strongly-convex, \( L \)-smooth function and \( \oracle{} \) a SFO such that \( \rbr{f, \oracle{}} \) satisfies the strong growth condition with parameter \( \rho \).
    Then stochastic gradient descent with fixed step-size \( \eta \leq \frac{2}{\rho(\mu + L)} \) converges as 
    \[ \E\sbr{f(\wk)} - f(\wopt) \leq \rbr{\frac{L}{\mu}}\rbr{1 - \frac{2\eta \mu L}{\mu + L}}^k \rbr{f(\w_0) - f(\wopt)}. \] 
\end{restatable}%
\noindent See \autoref{app:sgd-sc} for proof.\hfill \break

We compare this convergence rate to the original result given by \citet[Theorem 5]{vaswani2019fast} in \autoref{table:sgd-comparison}.
Our analysis allows for a larger step-size and establishes asymptotically faster convergence.
The improvement is most significant for ill-conditioned problems, where \( \mu \ll L \) implies \( \frac{2 L}{\mu + L} \gg 1 \). 
Finally, this result is tight in the sense that when \( \rho = 1 \) (which holds in the deterministic setting), it recovers the best known convergence rate for gradient descent on strongly-convex functions~\citep[Theorem 3.12]{bubeck2015convex}.

A key feature of \autoref{thm:sgc-convex} is that it requires only the full function \( f \) to be strongly-convex; the stochastic functions \( f(\cdot, \z) \) may be merely convex, as is typically the case in the finite-sum setting.
To illustrate this, consider the over-determined linear regression problem
\[ \min_{\w \in R^d} \sum_{i=1}^n \rbr{\abr{\w, x_i} - y_i}^2, \]
where \( d \ll n \).
The individual functions \( f_i(\w) = \rbr{\abr{\w, x_i} - y_i}^2 \) are clearly not strongly-convex,\footnote{To see this, consider the points \(w\) and \( \w' = \w + v \), where \( v \) is orthogonal to \( x_i \).} but \( f \) is as long as the data-matrix \( X \) is full-rank. 
However, in the unlikely case that if \oracle{} is also individually-strongly-convex, we can show that SGD will converge almost surely, as established in the following theorem.

\begin{restatable}{theorem}{sgcIndSC}~\label{thm:sgc-ind-sc}
    Let \( f \) be a \( \mu \)-strongly-convex, \( L \)-smooth function and \( \oracle{} \) an \( \Lmax \) individually-smooth, \( \mumax \)-individually-strongly-convex SFO such that \( \rbr{f, \oracle{}} \) satisfies minimizer interpolation. 
    Then stochastic gradient descent with fixed step-size \( \eta \leq \frac{2}{\mumax + \Lmax} \) converges deterministically at the rate 
    \[ f(\wkk) - f(\wopt) \leq \frac{L}{\mu} \rbr{1 - 2 \eta \, \delta_{\text{max}} }^k \rbr{f(\w_0) - f(\wopt)}, \] 
    where \( \delta_{\text{max}} = \max_{\z \in \calZ} \frac{\mu_\z L_\z}{\mu_\z + L_\z} \).
\end{restatable}%
\noindent See \autoref{app:sgd-sc} for proof.\hfill \break

This result is not surprising; strongly-convex functions have only a single minimizer, meaning that gradient-descent on a single sub-function \( f_i \) is sufficient to recover the global minimizer in the finite-sum setting. 
The bound in \autoref{thm:sgc-ind-sc} is also clearly sub-optimal for general stochastic optimization problems where \oracle{} is individually-strongly-convex and the optimization procedure has direct access to \( f(\cdot, \z ) \) for each \( \z \in \calZ \).
In this case, gradient descent on the best-conditioned stochastic function will converge to the global minimizer of the full function \( f \) as \( O\rbr{\exp\cbr{ - 2 \eta \, \delta_{\text{min}} \, K}} \), where \( \delta_{\text{min}} = \min_{\z \in \calZ} \frac{\mu_z L_\z}{\mu_z + L_\z} \)~\citep{bubeck2015convex}.
Notice that we have exchanged worst-case performance for best-case (\( \delta_{\text{max}} \) vs \( \delta_{\text{min}} \)) by optimizing \( f(\cdot, \z) \) directly. 
This illustrates the dangers of assuming individual strong-convexity and interpolation hold simultaneously.


\section{Convergence for Convex Functions}~\label{sec:sgd-convex}

From the perspective of the optimizer, convex functions are significantly more interesting than strongly-convex functions when interpolation is satisfied. 
Minimizer interpolation for convex functions implies \( \calX^* \subseteq \calX^*_{\z} \) --- the optimal set for \( f \) is a subset of the optimal set for \( f(\cdot, \z) \).
This condition intuitively feels milder than the requirement for a shared optimal point \( \wopt \). 
Moreover, assuming individual convexity does not lead to degenerate optimization problems such as those seen in the previous section.
Convex functions also require a major shift in analysis; now we shall show concentration of the optimality gap \( f(\w) - f(\wopt) \), rather than shrinking distance to a specific minimizer, \( \norm{\w - \wopt}^2 \).

We first establish the convergence rate of SGD for convex functions under the weak growth condition. 
As before, it is strictly more general to analyze the complexity of SGD under weak growth than in the setting where \oracle{} is individually-smooth and minimizer interpolation holds. 
Specifically, \autoref{lemma:interpolation-to-wgc} guarantees \( \rho \leq \frac{\Lmax}{L} \) in such a case. 
The following result improves on that given by \citet{vaswani2019fast} by constant factors and allows for a larger step-size (see \autoref{table:sgd-comparison}).  
Moreover, the proof in \autoref{app:sgd-convex} is simpler and far shorter. 

\begin{restatable}{theorem}{wgcConvex}~\label{thm:wgc-convex}
    Let \( f \) be a convex, \( L \)-smooth function and \oracle{} a SFO such that \( \rbr{f, \oracle{}} \) satisfies the weak growth condition with parameter \( \rho \).
    Then stochastic gradient descent with fixed step-size \( \eta < \frac{1}{\rho L} \) converges as
    \[ \E\sbr{\f(\bar \w_K)} - f(\wopt) \leq \frac{1}{2 \eta \rbr{1 - \eta \rho L} \, K} \norm{\w_0 - \wopt}^2, \]
    where \( \bar \w_K = \frac{1}{K} \sum_{k=0}^{K-1} \wk \). 
\end{restatable}

In fact, a slightly larger step-size and faster convergence rate can be obtained when \( \oracle{} \) is individually smooth.
We show this now.

\begin{restatable}{theorem}{wgcConvexIndSmooth}~\label{thm:wgc-convex-ind-smooth}
    Let \( f \) be a convex, \( L \)-smooth function and \oracle{} a SFO such that \( \rbr{f, \oracle{}} \) satisfies the weak growth condition with parameter \( \rho \).
    Moreover, suppose \oracle{} is \( \Lmax \) individually-smooth. 
    Then stochastic gradient descent with fixed step-size \( \eta < \frac{1}{\rho L} + \frac{1}{\Lmax} \) converges as
    \[ \E\sbr{f(\bar w_K)} - f(\wopt) \leq \frac{1}{2\eta \, \delta \, K} \norm{\w_0 - \wopt}^2,   \]
    where \( \bar \w_K = \frac{1}{K} \sum_{k=0}^{K-1} \wk \) and \( \delta = \min \cbr{ 1, 1 + \rho L \rbr{\frac{1}{\Lmax} - \eta }} \). 
\end{restatable}
\noindent See \autoref{app:sgd-convex} for proof. \hfill \break

\autoref{thm:wgc-convex-ind-smooth} is tight with the deterministic case in the following sense: if \( f(\cdot, \z) = f \) for each \( z \in \calZ \), then \( \rho = 1 \), \( \Lmax = L \), and the rate given is comparable to the best known results in the deterministic setting (cf. \citet[Theorem 3.3]{bubeck2015convex}). 
This result also further illustrates the benefits of directly assuming the weak growth condition, since the maximum step-size satisfies 
\( \frac{1}{\rho L} + \frac{1}{\Lmax} \geq \frac{2}{\Lmax}, \)
where \( \frac{2}{\Lmax} \) is the condition obtained when deriving weak growth directly from individual smoothness and minimizer interpolation.
\todo{Cannot directly compare the strongly-convex cases because fast and faster uses weak growth. Mismatch in \( \rho \). }
\begin{table}[t]
    \centering
    \begin{tabular}{c l l c c  }\toprule
        \multirow{2}{*}{Assumptions} & \multicolumn{2}{c}{Max. Step-Size}  & \multicolumn{2}{c}{Convergence Rate}\\%
        \cmidrule(lr){2-3} \cmidrule(l){4-5}
                 & \multicolumn{1}{c}{Ours} & \multicolumn{1}{c}{VBS~\citep{vaswani2019fast}} & \multicolumn{1}{c}{Ours} & \multicolumn{1}{c}{VBS~\citep{vaswani2019fast}}\\ \midrule
    \( \mu \)-SC & \( \eta \leq \frac{2}{\rho\rbr{\mu + L}} \)% 
                 & \( \eta \leq \frac{1}{\rho L} \)%
                 & \( O\rbr{\exp\cbr{- \rbr{\frac{2 \eta \mu L}{\mu + L}} \, K}} \)% 
                 & \( O\rbr{\exp\cbr{- \eta \mu \, K }} \) \\ \addlinespace
    Convex       & \( \eta < \frac{1}{\rho L} \)%
                 & \( \eta < \frac{1}{2 \rho L} \)%
                 & \( O\rbr{\frac{1}{2 \eta (1 - \eta \rho L) \, K}} \)%
                 & \( O\rbr{\frac{2 - \eta L\rbr{1 - 2 \eta \rho L}}{\eta (1- 2 \eta \rho L) \, K} } \)\\ \addlinespace 
    \makecell{Convex + \\ Ind. Smooth}% 
                 & \( \eta \leq \frac{1}{\rho L} + \frac{1}{\Lmax} \)%
                 & \multicolumn{1}{c}{N/A}% 
                 & \( O\rbr{\frac{1}{2 \eta \, \delta \, K}} \)%
                 & \multicolumn{1}{c}{N/A} \\ \bottomrule 
    \end{tabular}
    \caption{Comparison of convergence rates for fixed step-size SGD. Recall from \autoref{thm:wgc-convex-ind-smooth} that \( \delta = \min \cbr{ 1, 1 + \rho L \rbr{\frac{1}{\Lmax} - \eta }} \). Our results are tighter than VBS (Vaswani-Bach-Schmidt~\citep{vaswani2019fast}) and allow for larger step-sizes.}%
    \label{table:sgd-comparison}
\end{table}


\section{Almost Sure Convergence}~\label{sec:almost-sure}

Now we briefly change paradigms and consider the asymptotic behavior of SGD with a fixed step-size.
Our overall goal in this section is to show the almost-sure (with probability 1) convergence of SGD to a minimizer or stationary point of \( f \) when \( \rbr{f, \oracle{}} \) satisfy weak or strong growth, respectively.
In particular, we want to show that the random variable \( \lim_{\iter \rightarrow \infty} \norm{\grad(\wk)}^2 \) exists and is almost-surely zero.
We will need some tools from measure-theoretic probability to accomplish this.
To start, we note that each iterate \( \w_k \) can be written as deterministic Borel function of the stochastic gradients \( \cbr{\grad(\wk, \zk)}_{\iter=0}^{K} \) by unrolling the SGD update.
Formally, we also assume a probability space \( \rbr{\Omega,  \calF, P} \) in the background and say the sequence \( \seq{\wk} \) is \emph{adapted} to the filtration generated by the stochastic gradients,
\begin{align*}
    \calF_K = \sigma \rbr{\bigcup_{\iter = 0}^{K-1} \sigma \grad(\wk, \zk)}.
\end{align*}
The sequences of function and gradient values are functions of \( \seq{\wk} \) and so are also adapted to \( \seq{\calF_\iter} \).
See \citet{ccinlar2011probability} additional details on filtrations.

Our main tool to show convergence of sequences of random variables will be supermartingale theory~\citep{ccinlar2011probability}.
Supermartingales are one of two classic tools used analyze the convergence of SGD, the other being Lyapunov functions~\citep{bertsekas2000gradient}.
In particular, recent authors have made use of convergence of discrete-time, positive supermartingales~\citep{bertsekas2011incremental, nguyen2018sgd}.
This theorem is due to~\citet{neveu1975discrete} and will be the cornerstone of our analyses; we reproduce it here for convenience.

\begin{theorem}[Convergence of Positive Supermartingales]\label{thm:positive_supermartingales}
    Let \( \seq{Y_\iter} \), \( \seq{X_\iter} \), and \( \seq{A_\iter} \) be discrete, non-negative random processes indexed by \( \iter \in \bbN \) and adapted to the filtration \( \seq{\calF_\iter} \).
    Suppose that
    \begin{align*}
        \forall \iter \in \bbN, \: \E \sbr{Y_{\iter+1} \mid \calF_\iter} \leq Y_\iter - X_\iter + A_\iter,
        && \text{ and } &&
        \sum_{k=0}^{\infty} A_\iter < \infty,
    \end{align*}
    almost surely.
    Then the sequence \( \seq{Y_\iter} \) converges almost surely to a non-negative random variable \( Y_\infty \) and \( \sum_{k=0}^{\infty} X_\iter < \infty \) almost surely.
\end{theorem}
With necessary the measure-theoretic background complete, we are now ready to study the almost-sure convergence of stochastic gradient descent.

\begin{restatable}{theorem}{wgcAlmostSure}~\label{thm:wgc-almost-sure}
    Let \( \f \) be a convex, \( L \)-smooth function with at least one finite minimizer and \oracle{} a \( \Lmax \) individually-smooth SFO such that \( \rbr{f, \oracle{}} \) satisfy the weak growth condition with parameter \( \rho \).
    Then stochastic gradient descent with fixed step-size \( \eta < \frac{1}{\rho L} + \frac{1}{\Lmax} \) converges to a minimizer of \( f \) almost surely,
\end{restatable}

The proof is given in \autoref{app:almost-sure-convergence} and follows a straightforward argument.
First, we establish that the sequence of distances to a finite minimizer \( \seq{\norm{\wk - \wopt}^2} \) satisfies the conditions of Theorem~\ref{thm:positive_supermartingales}.
As by-product, we obtain that the series \( \sum_{k=0}^{\infty} \rbr{f(\wk) - f(\wopt)} \) is convergent and then deduce
\[ \lim_{k\rightarrow\infty} \wk \in \argmin_w f(\w), \] 
almost surely using the weak growth condition.
It is straightforward to prove an alternative version of \autoref{thm:wgc-almost-sure} which does not require individual smoothness.
In this case, convergence is established for \( \eta < \frac{1}{\rho L} \) using the same progress condition as in \autoref{thm:wgc-convex}, namely \autoref{eq:cwg-alternate-progress}.

\autoref{thm:wgc-almost-sure} holds for the \emph{final} iterate generated by the SGD procedure.
This should be contrasted with with Theorems~\ref{thm:wgc-convex} and~\ref{thm:wgc-convex-ind-smooth}, which apply only to the averaged iterate \( \bar \wk \).
While the existence of an asymptotic result suggests that non-asymptotic, final-iterate convergence for constant step-size SGD under the weak growth condition is possible, we do not establish such a result in this work. 
Convergence (or non-convergence) of the final iterate remains an interesting and surprisingly simple open problem in optimization under interpolation.

The final result of this chapter shows almost-sure convergence to a stationary point for general non-convex functions \( f \) such that \( \rbr{f, \oracle{}} \) satisfy the strong growth condition.
The proof is presented in \autoref{app:almost-sure-convergence} and follows a similar structure to the proof of \autoref{thm:wgc-almost-sure}.

\begin{restatable}{theorem}{sgcAlmostSure}~\label{thm:sgc-almost-sure}
    Let \( \f \) be an \( L \)-smooth function satisfying the strong growth condition with parameter \(\rho \).
    Then stochastic gradient descent with fixed step-size \(\eta < \frac{2}{\rho L} \) converges to a stationary point of \( f \) almost surely.
\end{restatable}

\endinput

