%%%%%%%%%%%%%%%%%%%%%%%%%%%%%%%%%%%%%%%%%%%%%%%%%%%%%%%%%%%%%%%%%%%%%%
% Template for a UBC-compliant dissertation
% At the minimum, you will need to change the information found
% after the "Document meta-data"
%
%!TEX TS-program = pdflatex
%!TEX encoding = UTF-8 Unicode

%% The ubcdiss class provides several options:
%%   gpscopy (aka fogscopy)
%%       set parameters to exactly how GPS specifies
%%         * single-sided
%%         * page-numbering starts from title page
%%         * the lists of figures and tables have each entry prefixed
%%           with 'Figure' or 'Table'
%%       This can be tested by `\ifgpscopy ... \else ... \fi'
%%   10pt, 11pt, 12pt
%%       set default font size
%%   oneside, twoside
%%       whether to format for single-sided or double-sided printing
%%   balanced
%%       when double-sided, ensure page content is centred
%%       rather than slightly offset (the default)
%%   singlespacing, onehalfspacing, doublespacing
%%       set default inter-line text spacing; the ubcdiss class
%%       provides \textspacing to revert to this configured spacing
%%   draft
%%       disable more intensive processing, such as including
%%       graphics, etc.
%%

% For submission to GPS
\documentclass[gpscopy,onehalfspacing,11pt]{ubcdiss}

% For your own copies (looks nicer)
% \documentclass[balanced,twoside,11pt]{ubcdiss}

%%%%%%%%%%%%%%%%%%%%%%%%%%%%%%%%%%%%%%%%%%%%%%%%%%%%%%%%%%%%%%%%%%%%%%
%%%%%%%%%%%%%%%%%%%%%%%%%%%%%%%%%%%%%%%%%%%%%%%%%%%%%%%%%%%%%%%%%%%%%%
%%
%% FONTS:
%%
%% The defaults below configures Times Roman for the serif font,
%% Helvetica for the sans serif font, and Courier for the
%% typewriter-style font.  Configuring fonts can be time
%% consuming; we recommend skipping to END FONTS!
%%
%% If you're feeling brave, have lots of time, and wish to use one
%% your platform's native fonts, see the commented out bits below for
%% XeTeX/XeLaTeX.  This is not for the faint at heart.
%% (And shouldn't you be writing? :-)
%%

%% NFSS font specification (New Font Selection Scheme)
%% \usepackage{times,mathptmx,courier}
%% \usepackage[scaled=.92]{helvet}

%% Math or theory people may want to include the handy AMS macros
\usepackage{amssymb}
\usepackage{amsmath}
\usepackage{amsfonts}

%% The pifont package provides access to the elements in the dingbat font.
%% Use \ding{##} for a particular dingbat (see p7 of psnfss2e.pdf)
%%   Useful:
%%     51,52 different forms of a checkmark
%%     54,55,56 different forms of a cross (saltyre)
%%     172-181 are 1-10 in open circle (serif)
%%     182-191 are 1-10 black circle (serif)
%%     192-201 are 1-10 in open circle (sans serif)
%%     202-211 are 1-10 in black circle (sans serif)
%% \begin{dinglist}{##}\item... or dingautolist (which auto-increments)
%% to create a bullet list with the provided character.
\usepackage{pifont}

%%%%%%%%%%%%%%%%%%%%%%%%%%%%%%%%%%%%%%%%%%%%%%%%%%%%%%%%%%%%%%%%%%%%%%
%% Configure fonts for XeTeX / XeLaTeX using the fontspec package.
%% Be sure to check out the fontspec documentation.
%\usepackage{fontspec,xltxtra,xunicode}	% required
%\defaultfontfeatures{Mapping=tex-text}	% recommended
%% Minion Pro and Myriad Pro are shipped with some versions of
%% Adobe Reader.  Adobe representatives have commented that these
%% fonts can be used outside of Adobe Reader.
%\setromanfont[Numbers=OldStyle]{Minion Pro}
%\setsansfont[Numbers=OldStyle,Scale=MatchLowercase]{Myriad Pro}
%\setmonofont[Scale=MatchLowercase]{Andale Mono}

%% Other alternatives:
%\setromanfont[Mapping=tex-text]{Adobe Caslon}
%\setsansfont[Scale=MatchLowercase]{Gill Sans}
%\setsansfont[Scale=MatchLowercase,Mapping=tex-text]{Futura}
%\setmonofont[Scale=MatchLowercase]{Andale Mono}
%\newfontfamily{\SYM}[Scale=0.9]{Zapf Dingbats}
%% END FONTS
%%%%%%%%%%%%%%%%%%%%%%%%%%%%%%%%%%%%%%%%%%%%%%%%%%%%%%%%%%%%%%%%%%%%%%
%%%%%%%%%%%%%%%%%%%%%%%%%%%%%%%%%%%%%%%%%%%%%%%%%%%%%%%%%%%%%%%%%%%%%%



%%%%%%%%%%%%%%%%%%%%%%%%%%%%%%%%%%%%%%%%%%%%%%%%%%%%%%%%%%%%%%%%%%%%%%
%%%%%%%%%%%%%%%%%%%%%%%%%%%%%%%%%%%%%%%%%%%%%%%%%%%%%%%%%%%%%%%%%%%%%%
%%
%% Recommended packages
%%
\usepackage{checkend}	% better error messages on left-open environments
\usepackage{graphicx}	% for incorporating external images

%% booktabs: provides some special commands for typesetting tables as used
%% in excellent journals.  Ignore the examples in the Lamport book!
\usepackage{booktabs}

%% listings: useful support for including source code listings, with
%% optional special keyword formatting.  The \lstset{} causes
%% the text to be typeset in a smaller sans serif font, with
%% proportional spacing.
\usepackage{listings}
\lstset{basicstyle=\sffamily\scriptsize,showstringspaces=false,fontadjust}

%% The acronym package provides support for defining acronyms, providing
%% their expansion when first used, and building glossaries.  See the
%% example in glossary.tex and the example usage throughout the example
%% document.
%% NOTE: to use \MakeTextLowercase in the \acsfont command below,
%%   we *must* use the `nohyperlinks' option -- it causes errors with
%%   hyperref otherwise.  See Section 5.2 in the ``LaTeX 2e for Class
%%   and Package Writers Guide'' (clsguide.pdf) for details.
\usepackage[printonlyused,nohyperlinks]{acronym}
%% The ubcdiss.cls loads the `textcase' package which provides commands
%% for upper-casing and lower-casing text.  The following causes
%% the acronym package to typeset acronyms in small-caps
%% as recommended by Bringhurst.
\renewcommand{\acsfont}[1]{{\scshape \MakeTextLowercase{#1}}}

%% color: add support for expressing colour models.  Grey can be used
%% to great effect to emphasize other parts of a graphic or text.
%% For an excellent set of examples, see Tufte's "Visual Display of
%% Quantitative Information" or "Envisioning Information".
\usepackage{color}
\definecolor{greytext}{gray}{0.5}

%% comment: provides a new {comment} environment: all text inside the
%% environment is ignored.
%%   \begin{comment} ignored text ... \end{comment}
\usepackage{comment}

%% The natbib package provides more sophisticated citing commands
%% such as \citeauthor{} to provide the author names of a work,
%% \citet{} to produce an author-and-reference citation,
%% \citep{} to produce a parenthetical citation.
%% We use \citeeg{} to provide examples
\usepackage[numbers,sort&compress]{natbib}
\newcommand{\citeeg}[1]{\citep[e.g.,][]{#1}}

%% The titlesec package provides commands to vary how chapter and
%% section titles are typeset.  The following uses more compact
%% spacings above and below the title.  The titleformat that follow
%% ensure chapter/section titles are set in singlespace.
\usepackage[compact]{titlesec}
\titleformat*{\section}{\singlespacing\raggedright\bfseries\Large}
\titleformat*{\subsection}{\singlespacing\raggedright\bfseries\large}
\titleformat*{\subsubsection}{\singlespacing\raggedright\bfseries}
\titleformat*{\paragraph}{\singlespacing\raggedright\itshape}

%% The caption package provides support for varying how table and
%% figure captions are typeset.
\usepackage[format=hang,indention=-1cm,labelfont={bf},margin=1em]{caption}

%% url: for typesetting URLs and smart(er) hyphenation.
%% \url{http://...}
\usepackage{url}
\urlstyle{sf}	% typeset urls in sans-serif

%% New Package Imports 

% math packages 
\usepackage{mathtools}
\usepackage{xfrac}
\usepackage{amsthm}
\usepackage{centernot}
\usepackage{bbm}

% writing packages
\usepackage[chapter]{easy-todo}

%% **TEMPORARY** %%
\usepackage{lineno}
\linenumbers

%%%%%%%%%%%%%%%%%%%%%%%%%%%%%%%%%%%%%%%%%%%%%%%%%%%%%%%%%%%%%%%%%%%%%%
%%%%%%%%%%%%%%%%%%%%%%%%%%%%%%%%%%%%%%%%%%%%%%%%%%%%%%%%%%%%%%%%%%%%%%
%%
%% Possibly useful packages: you may need to explicitly install
%% these from CTAN if they aren't part of your distribution;
%% teTeX seems to ship with a smaller base than MikTeX and MacTeX.
%%
%\usepackage{pdfpages}	% insert pages from other PDF files
%\usepackage{longtable}	% provide tables spanning multiple pages
%\usepackage{chngpage}	% support changing the page widths on demand
%\usepackage{tabularx}	% an enhanced tabular environment

%% enumitem: support pausing and resuming enumerate environments.
%\usepackage{enumitem}

%% rotating: provides two environments, sidewaystable and sidewaysfigure,
%% for typesetting tables and figures in landscape mode.
%\usepackage{rotating}

%% subfig: provides for including subfigures within a figure,
%% and includes being able to separately reference the subfigures.
%\usepackage{subfig}

%% ragged2e: provides several new new commands \Centering, \RaggedLeft,
%% \RaggedRight and \justifying and new environments Center, FlushLeft,
%% FlushRight and justify, which set ragged text and are easily
%% configurable to allow hyphenation.
%\usepackage{ragged2e}

%% The ulem package provides a \sout{} for striking out text and
%% \xout for crossing out text.  The normalem and normalbf are
%% necessary as the package messes with the emphasis and bold fonts
%% otherwise.
%\usepackage[normalem,normalbf]{ulem}    % for \sout

%%%%%%%%%%%%%%%%%%%%%%%%%%%%%%%%%%%%%%%%%%%%%%%%%%%%%%%%%%%%%%%%%%%%%%
%% HYPERREF:
%% The hyperref package provides for embedding hyperlinks into your
%% document.  By default the table of contents, references, citations,
%% and footnotes are hyperlinked.
%%
%% Hyperref provides a very handy command for doing cross-references:
%% \autoref{}.  This is similar to \ref{} and \pageref{} except that
%% it automagically puts in the *type* of reference.  For example,
%% referencing a figure's label will put the text `Figure 3.4'.
%% And the text will be hyperlinked to the appropriate place in the
%% document.
%%
%% Generally hyperref should appear after most other packages

%% The following puts hyperlinks in very faint grey boxes.
%% The `pagebackref' causes the references in the bibliography to have
%% back-references to the citing page; `backref' puts the citing section
%% number.  See further below for other examples of using hyperref.
%% 2009/12/09: now use `linktocpage' (Jacek Kisynski): GPS now prefers
%%   that the ToC, LoF, LoT place the hyperlink on the page number,
%%   rather than the entry text.
\usepackage[bookmarks,bookmarksnumbered,%
    allbordercolors={0.8 0.8 0.8},%
    pagebackref,linktocpage%
    ]{hyperref}
%% The following change how the the back-references text is typeset in a
%% bibliography when `backref' or `pagebackref' are used
%%
%% Change \nocitations if you'd like some text shown where there
%% are no citations found (e.g., pulled in with \nocite{xxx})
\newcommand{\nocitations}{\relax}
%%\newcommand{\nocitations}{No citations}
%%
%\renewcommand*{\backref}[1]{}% necessary for backref < 1.33
\renewcommand*{\backrefsep}{,~}%
\renewcommand*{\backreftwosep}{,~}% ', and~'
\renewcommand*{\backreflastsep}{,~}% ' and~'
\renewcommand*{\backrefalt}[4]{%
\textcolor{greytext}{\ifcase #1%
\nocitations%
\or
\(\rightarrow\) page #2%
\else
\(\rightarrow\) pages #2%
\fi}}


%% The following uses most defaults, which causes hyperlinks to be
%% surrounded by colourful boxes; the colours are only visible in
%% PDFs and don't show up when printed:
%\usepackage[bookmarks,bookmarksnumbered]{hyperref}

%% The following disables the colourful boxes around hyperlinks.
%\usepackage[bookmarks,bookmarksnumbered,pdfborder={0 0 0}]{hyperref}

%% The following disables all hyperlinking, but still enabled use of
%% \autoref{}
%\usepackage[draft]{hyperref}

%% The following commands causes chapter and section references to
%% uppercase the part name.
\renewcommand{\chapterautorefname}{Chapter}
\renewcommand{\sectionautorefname}{Section}
\renewcommand{\subsectionautorefname}{Section}
\renewcommand{\subsubsectionautorefname}{Section}

%% If you have long page numbers (e.g., roman numbers in the
%% preliminary pages for page 28 = xxviii), you might need to
%% uncomment the following and tweak the \@pnumwidth length
%% (default: 1.55em).  See the tocloft documentation at
%% http://www.ctan.org/tex-archive/macros/latex/contrib/tocloft/
% \makeatletter
% \renewcommand{\@pnumwidth}{3em}
% \makeatother

%%%%%%%%%%%%%%%%%%%%%%%%%%%%%%%%%%%%%%%%%%%%%%%%%%%%%%%%%%%%%%%%%%%%%%
%%%%%%%%%%%%%%%%%%%%%%%%%%%%%%%%%%%%%%%%%%%%%%%%%%%%%%%%%%%%%%%%%%%%%%
%%
%% Some special settings that controls how text is typeset
%%
% \raggedbottom		% pages don't have to line up nicely on the last line
% \sloppy		% be a bit more relaxed in inter-word spacing
% \clubpenalty=10000	% try harder to avoid orphans
% \widowpenalty=10000	% try harder to avoid widows
% \tolerance=1000

%% And include some of our own useful macros
% This file provides examples of some useful macros for typesetting
% dissertations.  None of the macros defined here are necessary beyond
% for the template documentation, so feel free to change, remove, and add
% your own definitions.
%
% We recommend that you define macros to separate the semantics
% of the things you write from how they are presented.  For example,
% you'll see definitions below for a macro \file{}: by using
% \file{} consistently in the text, we can change how filenames
% are typeset simply by changing the definition of \file{} in
% this file.
% 
%% The following is a directive for TeXShop to indicate the main file
%%!TEX root = main.tex

\newcommand{\NA}{\textsc{n/a}}	% for "not applicable"
\newcommand{\eg}{e.g.,\ }	% proper form of examples (\eg a, b, c)
\newcommand{\ie}{i.e.,\ }	% proper form for that is (\ie a, b, c)
\newcommand{\etal}{\emph{et al}}

% Some useful macros for typesetting terms.
\newcommand{\file}[1]{\texttt{#1}}
\newcommand{\class}[1]{\texttt{#1}}
\newcommand{\latexpackage}[1]{\href{http://www.ctan.org/macros/latex/contrib/#1}{\texttt{#1}}}
\newcommand{\latexmiscpackage}[1]{\href{http://www.ctan.org/macros/latex/contrib/misc/#1.sty}{\texttt{#1}}}
\newcommand{\env}[1]{\texttt{#1}}
\newcommand{\BibTeX}{Bib\TeX}

% Define a command \doi{} to typeset a digital object identifier (DOI).
% Note: if the following definition raise an error, then you likely
% have an ancient version of url.sty.  Either find a more recent version
% (3.1 or later work fine) and simply copy it into this directory,  or
% comment out the following two lines and uncomment the third.
\DeclareUrlCommand\DOI{}
\newcommand{\doi}[1]{\href{http://dx.doi.org/#1}{\DOI{doi:#1}}}
%\newcommand{\doi}[1]{\href{http://dx.doi.org/#1}{doi:#1}}

% Useful macro to reference an online document with a hyperlink
% as well with the URL explicitly listed in a footnote
% #1: the URL
% #2: the anchoring text
\newcommand{\webref}[2]{\href{#1}{#2}\footnote{\url{#1}}}

% epigraph is a nice environment for typesetting quotations
\makeatletter
\newenvironment{epigraph}{%
	\begin{flushright}
	\begin{minipage}{\columnwidth-0.75in}
	\begin{flushright}
	\@ifundefined{singlespacing}{}{\singlespacing}%
    }{
	\end{flushright}
	\end{minipage}
	\end{flushright}}
\makeatother

% \FIXME{} is a useful macro for noting things needing to be changed.
% The following definition will also output a warning to the console
\newcommand{\FIXME}[1]{\typeout{**FIXME** #1}\textbf{[FIXME: #1]}}

\usepackage{tcolorbox}
\newtcolorbox{commentboxone}{%
colback=red!5!white,%
colframe=red!75!black,%
left=.5em,%
right=.5em,%
top=.5em,%
bottom=.5em%
}
\newcommand{\fred}[1]{\begin{commentboxone}\footnotesize {\sc FK:} #1\end{commentboxone}}
% END

%% Include custom commands
% bf series
\def\bfA{\mathbf{A}}
\def\bfB{\mathbf{B}}
\def\bfC{\mathbf{C}}
\def\bfD{\mathbf{D}}
\def\bfE{\mathbf{E}}
\def\bfF{\mathbf{F}}
\def\bfG{\mathbf{G}}
\def\bfH{\mathbf{H}}
\def\bfI{\mathbf{I}}
\def\bfJ{\mathbf{J}}
\def\bfK{\mathbf{K}}
\def\bfL{\mathbf{L}}
\def\bfM{\mathbf{M}}
\def\bfN{\mathbf{N}}
\def\bfO{\mathbf{O}}
\def\bfP{\mathbf{P}}
\def\bfQ{\mathbf{Q}}
\def\bfR{\mathbf{R}}
\def\bfS{\mathbf{S}}
\def\bfT{\mathbf{T}}
\def\bfU{\mathbf{U}}
\def\bfV{\mathbf{V}}
\def\bfW{\mathbf{W}}
\def\bfX{\mathbf{X}}
\def\bfY{\mathbf{Y}}
\def\bfZ{\mathbf{Z}}

% bb series
\def\bbA{\mathbb{A}}
\def\bbB{\mathbb{B}}
\def\bbC{\mathbb{C}}
\def\bbD{\mathbb{D}}
\def\bbE{\mathbb{E}}
\def\bbF{\mathbb{F}}
\def\bbG{\mathbb{G}}
\def\bbH{\mathbb{H}}
\def\bbI{\mathbb{I}}
\def\bbJ{\mathbb{J}}
\def\bbK{\mathbb{K}}
\def\bbL{\mathbb{L}}
\def\bbM{\mathbb{M}}
\def\bbN{\mathbb{N}}
\def\bbO{\mathbb{O}}
\def\bbP{\mathbb{P}}
\def\bbQ{\mathbb{Q}}
\def\bbR{\mathbb{R}}
\def\bbS{\mathbb{S}}
\def\bbT{\mathbb{T}}
\def\bbU{\mathbb{U}}
\def\bbV{\mathbb{V}}
\def\bbW{\mathbb{W}}
\def\bbX{\mathbb{X}}
\def\bbY{\mathbb{Y}}
\def\bbZ{\mathbb{Z}}

% cal series
\def\calA{\mathcal{A}}
\def\calB{\mathcal{B}}
\def\calC{\mathcal{C}}
\def\calD{\mathcal{D}}
\def\calE{\mathcal{E}}
\def\calF{\mathcal{F}}
\def\calG{\mathcal{G}}
\def\calH{\mathcal{H}}
\def\calI{\mathcal{I}}
\def\calJ{\mathcal{J}}
\def\calK{\mathcal{K}}
\def\calL{\mathcal{L}}
\def\calM{\mathcal{M}}
\def\calN{\mathcal{N}}
\def\calO{\mathcal{O}}
\def\calP{\mathcal{P}}
\def\calQ{\mathcal{Q}}
\def\calR{\mathcal{R}}
\def\calS{\mathcal{S}}
\def\calT{\mathcal{T}}
\def\calU{\mathcal{U}}
\def\calV{\mathcal{V}}
\def\calW{\mathcal{W}}
\def\calX{\mathcal{X}}
\def\calY{\mathcal{Y}}
\def\calZ{\mathcal{Z}}


%% Theorem Environments %%
\usepackage{amsthm}
\usepackage{thmtools, thm-restate}
\declaretheorem{theorem}
\declaretheorem{lemma}
\declaretheorem{definition}
\declaretheorem{corollary}
\declaretheorem{example}

%% Floor and Ceiling %%
\usepackage{mathtools} % required for \DeclarePairedDelimeter

% almost surely equal:
\def\equas{\stackrel{\text{\rm\tiny a.s.}}{=}}
% independent
\newcommand{\ind}[0]{\perp \!\!\! \perp }

\DeclarePairedDelimiter{\ceilpair}{\lceil}{\rceil}
\DeclarePairedDelimiter{\floor}{\lfloor}{\rfloor}
\newcommand{\argdot}{{\,\vcenter{\hbox{\tiny$\bullet$}}\,}} %generic argument dot

\newcommand{\seq}[1]{\rbr{#1}}

% easy bracketing:
\newcommand{\rbr}[1]{\left(#1\right)}
\newcommand{\sbr}[1]{\left[#1\right]}
\newcommand{\cbr}[1]{\left\{#1\right\}}
\newcommand{\abr}[1]{\left\langle#1\right\rangle}

% Norms
\def\norm#1{\|#1\|}
\def\biggnorm#1{\bigg\|#1\bigg\|}
% Random Variable Norms:
\def\psitwo#1{\|#1\|_{\psi_2}}
\def\psione#1{\|#1\|_{\psi_1}}

% mid
\newcommand{\biggmid}{\bigg \vert }

% argmax/argmin
\def\argmax{\mathop{\rm arg\,max}}
\def\argmin{\mathop{\rm arg\,min}}

% General Symbols
\def\half{\frac 1 2}
\newcommand{\inv}[1]{\frac{1}{#1}}
\newcommand{\halved}[1]{\frac{#1}{2}}
\newcommand{\R}{\mathbb{R}}

\newcommand{\into}{\rightarrow}

% Gradient Descent Symbols
\newcommand{\oracle}{\mbox{\( \calO \)}}
\newcommand{\iter}{k}

\newcommand{\Lk}{L_{\zk}}
\newcommand{\Lmax}{L_{\text{max}}}
\newcommand{\mumax}{\mu_{\text{max}}}
\newcommand{\Lmin}{L_{\text{min}}}
\newcommand{\mumin}{\mu_{\text{min}}}

% iterates
\newcommand{\y}{y}
\newcommand{\yk}{y_k}
\newcommand{\ykk}{y_{k+1}}

\newcommand{\vk}{v_k}
\newcommand{\vkk}{v_{k+1}}

\newcommand{\w}{w}
\newcommand{\wk}{w_k}
\newcommand{\wkk}{w_{k+1}}
\newcommand{\wopt}{w^*}
\newcommand{\wbar}{\bar{w}}
% noise
\newcommand{\Z}{Z}
\newcommand{\z}{z}
\newcommand{\zk}{z_{k}}
\newcommand{\zkk}{z_{k+1}}
% step-sizes
\newcommand{\tetak}{{\tilde{\eta}}_{k}}
\newcommand{\etamin}{\eta_{\text{min}}}
\newcommand{\etamax}{\eta_{\text{max}}}
\newcommand{\etak}{\eta_k}
\newcommand{\etakk}{\eta_{k+1}}

\newcommand{\betak}{\beta_{k}}
\newcommand{\betakk}{\beta_{k+1}}
\newcommand{\alphak}{\alpha_{k}}
\newcommand{\alphakk}{\alpha_{k+1}}

\newcommand{\Ek}{\bbE_{\zk}}
\newcommand{\E}{\bbE}

% functions
\newcommand{\f}{f}
\newcommand{\fj}{f_i}
\newcommand{\fopt}{f^*}
% sub-sampled functions
\newcommand{\fk}{f_{i_k}}
\newcommand{\fkk}{f_{i_{(k+1)}}}
% gradients
\newcommand{\grad}{\nabla f}
% sub-sampled gradients
\newcommand{\gradk}{\nabla f_{i_k}}
\newcommand{\gradkk}{\nabla f_{i_{(k+1)}}}

%% Weak and strong growth constants
\newcommand{\sgc}{\rho}
\newcommand{\wgc}{\alpha}
%%%%%%%%%%%%%%%%%%%%%%%%%%%%%%%%%%%%%%%%%%%%%%%%%%%%%%%%%%

%%%%%%%%%%%%%%%%%%%%%%%%%%%%%%%%%%%%%%%%%%%%%%%%%%%%%%%%%%%%%%%%%%%%%%
%%%%%%%%%%%%%%%%%%%%%%%%%%%%%%%%%%%%%%%%%%%%%%%%%%%%%%%%%%%%%%%%%%%%%%
%%
%% Document meta-data: be sure to also change the \hypersetup information
%%

\title{Over-parameterization, Growth Conditions, and Stochastic Gradient Descent}
%\subtitle{If you want a subtitle}

\author{Aaron Mishkin}
\previousdegree{B.Sc., University of British Columbia, 2018}

% What is this dissertation for?
\degreetitle{Master of Science}

\institution{The University of British Columbia}
\campus{Vancouver}

\faculty{The Faculty of Science}
\department{Computer Science}
\submissionmonth{August}
\submissionyear{2020}

% details of your examining committee
\examiningcommittee{Mark Schmidt, Computer Science}{Supervisor}
\examiningcommittee{TBD, Computer Science}{Supervisory Committee Member}
\examiningcommittee{TBD, Computer Science}{Supervisory Committee Member}

% details of your supervisory committee
\supervisorycommittee{TBD, Computer Science}{Supervisory Committee Member}
\supervisorycommittee{TBD, Computer Science}{Supervisory Committee Member}

%% hyperref package provides support for embedding meta-data in .PDF
%% files
\hypersetup{
  pdftitle={Over-parameterization, Growth Conditions, and Stochastic Gradient Descent (DRAFT: \today)},
  pdfauthor={Aaron Mishkin},
  pdfkeywords={}
}

%%%%%%%%%%%%%%%%%%%%%%%%%%%%%%%%%%%%%%%%%%%%%%%%%%%%%%%%%%%%%%%%%%%%%%
%%%%%%%%%%%%%%%%%%%%%%%%%%%%%%%%%%%%%%%%%%%%%%%%%%%%%%%%%%%%%%%%%%%%%%
%%
%% The document content
%%

%% LaTeX's \includeonly commands causes any uses of \include{} to only
%% include files that are in the list.  This is helpful to produce
%% subsets of your thesis (e.g., for committee members who want to see
%% the dissertation chapter by chapter).  It also saves time by
%% avoiding reprocessing the entire file.
%\includeonly{intro,conclusions}
%\includeonly{discussion}

\begin{document}

%%%%%%%%%%%%%%%%%%%%%%%%%%%%%%%%%%%%%%%%%%%%%%%%%%
%% From Thesis Components: Traditional Thesis
%% <http://www.grad.ubc.ca/current-students/dissertation-thesis-preparation/order-components>

% Preliminary Pages (numbered in lower case Roman numerals)
%    1. Title page (mandatory)
\maketitle

%    2. Committee page (mandatory): lists supervisory committee and,
%    if applicable, the examining committee
\makecommitteepage

% ### **TEMPORARY** ###
\listoftodos
% ### **TEMPORARY** ###

%    3. Abstract (mandatory - maximum 350 words)
%% The following is a directive for TeXShop to indicate the main file
%!TEX root = ../main.tex

\chapter{Abstract}

Current machine learning practice requires solving huge-scale empirical risk minimization problems quickly and robustly. 
These problems are often highly under-determined and admit multiple solutions which exactly fit, or \emph{interpolate}, the training data. 
In such cases, stochastic gradient descent has been shown to converge without decreasing step-sizes or averaging, and can achieve the fast convergence of deterministic gradient methods. 
Recent work has further shown that stochastic acceleration and line-search methods for step-size selection are possible in this restricted setting.
Although pioneering, existing analyses for first-order methods under interpolation have two major weaknesses:
they are restricted to the finite-sum setting, and, secondly, they are not tight with the best deterministic rates. 
To address these issues, we extend the notion of interpolation to stochastic optimization problems with general, first-order oracles.
We use the proposed framework to analyze stochastic gradient descent with a fixed step-size and with an Armijo-type line-search, as well as Nesterov's accelerated gradient method with stochastic gradients.
In nearly all settings, we obtain faster convergence with a wider range of parameters. 
The improvement for stochastic Nesterov acceleration is comparable to dividing by the square-root of the condition number and addresses criticism that existing rates were not truly ``accelerated''. 
In the case of convex functions, our convergence rates for stochastic gradient descent --- both with and without the stochastic Armijo line-search --- recover the best-known rates in the deterministic setting.
We also provide a simple extension to \( \ell_2 \)-regularized minimization, which opens the path to proximal-gradient methods and non-smooth optimization under interpolation. 


%% Unused

% Our theoretical results are developed for general stochastic first-order oracles and apply to finite-sum functions with sub-sampled gradients as a special case. 
%For example, modern neural network architectures are powerful enough to memorize large datasets and achieve near zero loss on many common natural-image benchmarks. 

\cleardoublepage

%    4. Lay Summary (Effective May 2017, mandatory - maximum 150 words)
%% The following is a directive for TeXShop to indicate the main file
%!TEX root = ../main.tex

%% https://www.grad.ubc.ca/current-students/dissertation-thesis-preparation/preliminary-pages
%%
%% LAY SUMMARY Effective May 2017, all theses and dissertations must
%% include a lay summary.  The lay or public summary explains the key
%% goals and contributions of the research/scholarly work in terms that
%% can be understood by the general public. It must not exceed 150
%% words in length.

\chapter{Lay Summary}
\todo{Write lay summary}

A major trend in machine learning is the use of flexible models which can exactly fit large quantities of data. 
For example, deep learning approaches can ``memorize'' datasets, meaning they achieve nearly perfect predictions on the samples used to fit the model. 
In this case, we say that the model \emph{interpolates} the dataset.
Interpolating models are particularly interesting from an optimization perspective because they can be fit very quickly using stochastic gradient methods. 
This contrasts with general models, where stochastic gradient methods are notoriously slow. 
In this thesis, we develop a rigorous definition of interpolation and study the speed of stochastic gradient methods for interpolating models.  
Our approach is more general than existing analyses and covers standard model-fitting using a dataset as a special case. 
For many model classes, we show stochastic gradient methods permit a wider range of parameters and are faster than previously known when interpolation is satisfied. 

\cleardoublepage

%    5. Preface
%% The following is a directive for TeXShop to indicate the main file
%!TEX root = ../main.tex

\chapter{Preface}

\todo{Write preface}

\cleardoublepage

%    6. Table of contents (mandatory - list all items in the preliminary pages
%    starting with the abstract, followed by chapter headings and
%    subheadings, bibliographies and appendices)
\tableofcontents
\cleardoublepage	% required by tocloft package

%    7. List of tables (mandatory if thesis has tables)
\listoftables
\cleardoublepage	% required by tocloft package

%    8. List of figures (mandatory if thesis has figures)
\listoffigures
\cleardoublepage	% required by tocloft package

%    9. List of illustrations (mandatory if thesis has illustrations)
%   10. Lists of symbols, abbreviations or other (optional)

%   11. Glossary (optional)
%% The following is a directive for TeXShop to indicate the main file
%%!TEX root = main.tex

\chapter{Glossary}

\begin{acronym}[R-SAGD]
  \acro{SGD}[SGD]{stochastic gradient descent}
  \acro{SFO}[SFO]{stochastic first-order oracle}
  \acro{AGD}[AGD]{accelerated gradient descent}
  \acro{SAGD}[SAGD]{stochastic accelerated gradient descent}
  \acro{RSAGD}[R-SAGD]{reformulated stochastic accelerated gradient descent}
\end{acronym}
	% always input, since other macros may rely on it

\textspacing		% begin one-half or double spacing

%   12. Acknowledgements (optional)
%% The following is a directive for TeXShop to indicate the main file
%!TEX root = ../main.tex

\chapter{Acknowledgments}

TODO


%   13. Dedication (optional)

% Body of Thesis (not all sections may apply)
\mainmatter

\acresetall	% reset all acronyms used so far





%    1. Introduction
%% The following is a directive for TeXShop to indicate the main file
%!TEX root = ../main.tex

\chapter{Introduction}\label{ch:Introduction}

Stochastic first-order methods are the most popular optimization algorithms in modern machine learning.
In particular, \ac{SGD}~\citep{robbins1951sgd} and its adaptive variants~\citep{duchi2011adagrad, tieleman2012rmsprop, zeiler2012adadelta, kingma2015adam} are widely used in large-scale supervised learning, where they are frequently referred to as fundamental ``workhorse'' algorithms~\citep{qian2019improvedrates, assran2019sgpush, grosse2015scaling}. 
The main advantage of these methods for \edit{machine learning} is that they only use the gradient of a single or small sub-sample of training examples to update the model parameters at each iteration.
The computational cost of \ac{SGD} (and variants) is thus independent of the training set size and scales to very large datasets and models.
This property is also why stochastic first-order methods are the dominant approach to training highly expressive models, such as deep neural networks~\citep{zhang2017understanding, bengio2012practical} and non-parametric kernels~\citep{liang2018just, belkin2019datainterp}.

Despite their successes, stochastic first-order methods suffer from two well known problems. 
The step-size, momentum term, and other algorithmic hyper-parameters must be carefully tuned to obtain good performance~\citep{bengio2012practical, schaul2013no, li2019convergence, choi2019empirical}; and
they converge slowly compared to deterministic~\citep{nesterov2004lectures} or variance-reduced algorithms~\citep{leroux2012sag, johnson2013svrg, defazio2014saga} even when well-tuned.
The issue of hyper-parameter tuning for \ac{SGD} is the focus of intense research, with approaches ranging from adaptive step-sizes inspired by online learning~\citep{luo2019adabound, li2019convergence, orabona2017coin} to meta-learning procedures~\citep{baydin2018hypergradient, schraudolph1999local, sutton1992gain, almeida1998parameter, plagianakos2001learning, shao2000rates} and heuristics for online estimation~\citep{schaul2013no, rolinek2018l4, tan2016bb}.
In contrast, the slow convergence of first-order methods in the general stochastic setting cannot be improved, with tight lower-bounds showing \( \Theta(\epsilon^{-4}) \) complexity for convergence to a stationary point~\citep{drori2019complexity, arjevani2019lower}.
This compares poorly to deterministic gradient methods, which are \( \Theta(\epsilon^{-2}) \) for the same problem class~\citep{carmon2019lower}.

Increasing experimental and theoretical evidence shows that the slow optimization speed of stochastic first-order methods is mitigated when training over-parameterized models~\citep{ma2018power, arora2018overparameterization, zhou2019analysis}.
For example, variance-reduced algorithms typically underperform \ac{SGD} in this setting despite using increased memory or computation~\citep{defazio2019effectiveness, ma2018power}. 
A key property of over-parameterized problems is that they satisfy \emph{interpolation}, meaning the model can exactly fit all of the available training data~\citep{belkin2019datainterp}.
\edit{While this may seem strong, interpolation has been observed in practice for popular techniques such as boosting~\citep{schapire1997boosting}, kernel learning~\citep{belkin2019datainterp}, and deep learning with over-parameterized architectures~\citep{belkin2019reconciling, zhang2013gradient}}. 
\edit{Under additional assumptions, interpolation implies the second-moment of the stochastic gradients is bounded by a linear function of either the gradient-norm, or the optimality gap.
These properties are called the strong~\citep{solodov1998incremental, tseng1998incremental, schmidt2013fast} and weak~\citep{vaswani2019fast, bassily2018exponential} growth conditions, respectively.} 
A form of automatic reduction in gradient noise occurs~\citep{liu2020accelerating} when strong or weak growth is satisfied, explaining why variance reduction may be unnecessary. 

The connection between over-parameterization and fast stochastic optimization has led to a wave of interest in analyzing first-order methods under interpolation and weak/strong growth.
A number of works have shown that \ac{SGD} obtains the fast convergence rate of deterministic gradient methods for interpolating models~\citep{schmidt2013fast, bassily2018exponential, vaswani2019fast, cevher2018linear, jain2018accelerating}, while closely related research has established accelerated convergence rates under an additional requirement for convexity~\citep{liu2020accelerating, vaswani2019fast, jain2018accelerating}.
Sub-sampled second-order methods have also been explored and proved to have local linear convergence for specific function classes~\citep{meng2020fastandfurious}.
These positive results show the interpolation setting is restricted enough to break the \( \Omega\rbr{\epsilon^{-4}} \) barrier for stochastic first-order methods and attain the optimal \( \Theta(\epsilon^{-1}) \) complexity (up to problem-specific constants) \edit{for finding stationary points of} smooth, convex functions~\cite{nemirovsky1985optimal, arjevani2016iteration}. 

Despite these rapid advances, theoretical rates and practical performance for \ac{SGD} under interpolation still rely on carefully selected hyper-parameters.
A number of approaches from the general stochastic setting are rapidly being adopted to tackle this problem.
For instance, several works have considered a stochastic version of the Polyak step-size, which does not require knowledge of the smoothness or convexity parameters~\citep{loizou2020sps, berrada2019training}. 
Stochastic line-searches using the classic Armijo condition~\citep{armijo1966ls} have also been proposed and shown to obtain fast convergence under interpolation~\citep{vaswani2019painless}.
Very recently, adaptive variants of \ac{SGD} without momentum have analyzed both with and without the Armijo line-search~\citep{vaswani2020adaptive}.

This thesis analyzes a small, but core group of methods in stochastic optimization under interpolation.
The focus is on first-order algorithms; we consider constant step-size \ac{SGD}, \ac{SGD} with a stochastic Armijo line-search, and a version of Nesterov's accelerated gradient descent with stochastic gradients~\citep{nesterov2004lectures}. 
In nearly all cases, we show that existing analyses can be tightened to yield faster convergence rates with a larger range of step-sizes. 
In the case of acceleration, the improvement is comparable to taking the square-root of the condition number and addresses criticisms that previous analyses yield inferior rates to those of \ac{SGD} in some circumstances~\cite{liu2020accelerating}. 

The thesis is organized as follows: 
In \autoref{ch:interpolation-gc}, we formalize interpolation and the strong/weak growth conditions in the context of general stochastic oracles. 
Connections between interpolation, smoothness properties of the stochastic oracle, and growth of the stochastic gradients are derived.
\autoref{ch:sgd} analyzes the complexity of \ac{SGD} with a fixed step-size, drawing comparisons with previous work as well as convergence rates in the deterministic setting. 
Asymptotic convergence of SGD with a constant step-size to
\begin{inparaenum}[(i)]
\item stationary points of general non-convex functions, and
\item minimizers of convex functions 
\end{inparaenum}
is shown under the strong and weak growth conditions.
Chapters~\ref{ch:line-search} and~\ref{ch:acceleration} then address the convergence of \ac{SGD} with a stochastic Armijo line-search and stochastic accelerated gradient descent, respectively. 
Finally, \autoref{ch:beyond-interpolation} leaves the basic interpolation setting behind and considers structural minimization with interpolating functions as components. 
Convergence to an explicit neighborhood is derived for \( L_2 \)-regularized problems as a special case. 


\section{Preliminaries and Assumptions}\label{sec:setup}

%% Basic Assumptions
This work considers the problem of minimizing a continuous function \( f : \R^d \into \R \) under interpolation conditions.
We assume that \( f \) is bounded below with at least one finite minimizer.
That is, there exists \( \wopt \in \R^d \) such that 
\begin{align*}
    f(w) \geq f(\wopt) \quad \quad \forall w \in \R^d. 
\end{align*}
Additionally, \( f \) is required to be differentiable and \( L \)-smooth, meaning the gradient mapping \( \w \mapsto \grad(\w) \) is an \( L \)-Lipschitz function,
\[ \norm{\grad(\w) - \grad(v)} \leq L \norm{\w - v} \quad \quad \forall \, \w, v \in \R^d. \]
This is equivalent to the existence of the following quadratic upper-bound on \( f \):
\begin{align*}
    f(v) &\leq f(w) \abr{\grad(\w), v - \w} + \frac{L}{2}\norm{v - \w}^2 &\forall \, \w,v \in \R^d. \tag{\( L \)-Smoothness} 
    \intertext{At times, we will further assume that \( f \) is convex or \( \mu \)-strongly-convex,}
    f(v) &\geq f(\w) + \abr{\grad(\w), v - \w} &\forall \, \w,v \in \R^d, \tag{Convexity} \\
    f(v) &\geq f(\w) + \abr{\grad(\w), v - \w} + \frac{\mu}{2}\norm{v - \w}^2 &\forall \, \w,v \in \R^d. \tag{\( \mu \)-Strong-Convexity}
\end{align*}
Convexity is satisfied for many optimization problems occurring in machine learning, including linear and logistic regression. 

Convexity can be relaxed to a more limited property called invexity.
Formally, we say a \edit{differentiable} function \( f \) is invex if all stationary points are also global minimizers of \( f \)~\citep{ben1986invexity}, meaning
\[ \grad(\w) = 0 \implies f(\w) \leq f(v) \quad \quad \forall \, v \in \R^d.  \]
Invexity is formally weaker than convexity and follows immediately from the first-order conditions for convexity given above.
The analogue of strong-convexity for invex functions is the Polyak-Łojasiewicz (PL) condition~\citep{karimi2016linear}.
A function is said to be \( \mu \)-PL if
\[ \frac{1}{2 \mu} \norm{\grad(\w)}^2 \leq f(\w) - f(\wopt), \]
holds for all \( \w \in \R^d \).
Again, the Polyak-Łojasiewicz condition is weaker than strong convexity; a \( \mu \)-strongly-convex function is \( \mu \)-PL, but the converse does not hold.

\edit{
In several cases, it will be useful to interpret our results in the context of finite-sum functions.
The function \( f \) is said to be a finite-sum if it can be written as 
\[ f(\w) = \frac{1}{n} \sum_{i=0}^n f_i(\w),  \]
where the individual (or sub-) functions \( f_i : \R^d \rightarrow \R \) may be strongly-convex, convex, or non-convex depending on the problem.
Such objective functions arise naturally in empirical risk minimization, where \( f_i \) typically corresponds to a single training example \( (x_i, y_i) \) in a training set.
For example, the classic least-squares regression objective can be written as 
\[ f(\w) = \frac{1}{n} \sum_{i=0}^n \rbr{\abr{\w, x_i} - y_i}^2, \] 
which is finite-sum with sub-functions \( f_i(\w) =  \rbr{\abr{\w, x_i} - y_i}^2 \).
}

\section{Related Work}~\label{sec:related-work}

\noindent \textbf{Interpolation}:
Existing work focuses on interpolation in the context of finite-sum objectives. 
In this setting, \citet{bassily2018exponential} say an objective-optimizer pair satisfies interpolation if the evaluations of the sub-functions approach the same minimal value in the iteration limit. 
\citet{berrada2019training} propose an \( \epsilon \)-interpolation definition which requires the values of the sub-functions to be lower-bounded and within \( \epsilon  \) of that bound at the minimizer of the overall objective. 
A larger body of work considers interpolation to hold when the global minimizers of the overall objective are either global minimizers or stationary points of the sub-functions~\cite{vaswani2019fast, vaswani2019painless, vaswani2020adaptive, meng2020fastandfurious, loizou2020sps}.\\

\noindent \textbf{Growth Conditions}:
The strong growth condition was first proposed in the context of incremental gradient methods by \citet{solodov1998incremental} and \citet{tseng1998incremental} as an almost-sure bound on the squared-norm of the stochastic gradients. 
This definition was later used by~\citet{schmidt2013fast} to derive linear convergence rates for \ac{SGD} with a constant step-size. 
\citet{vaswani2019fast} propose a modified version of strong growth which holds in expectation, rather than almost-surely. 
We call this latter condition strong growth and refer to the earlier definition as maximal strong growth.
\citet{vaswani2019fast} also propose the weak growth condition as a natural relaxation of strong growth. 

\citet{cevher2018linear} refer to strong growth simply as ``the growth condition'' and suggest strong growth with an additive noise parameter as an alternative relaxation, which they also call weak growth.
For clarity, we refer to this assumption as strong growth with additive noise.
For unbiased stochastic oracles, this condition is equivalent to assuming a bound on the variance of the stochastic gradients~\citep{khaled2020better, ghadimi2012optimal1}.
\citet{cevher2018linear} study the convergence of proximal-gradient methods under this condition and also prove that strong growth is both sufficient \emph{and necessary} for \ac{SGD} to converge linearly.\\


\noindent \textbf{Stochastic Acceleration}: 

\todo{cite \citet{honorio2012biased}}

Many authors have considered accelerating stochastic gradient methods.
\citet{schmidt2011convergence} provide sufficient conditions on gradient noise for acceleration of proximal-gradient methods.
\edit{D'Aspremont}~\citep{aspremont2008approximate} and \citet{devolder2014first} investigate accelerated gradient methods under the assumption of approximate oracles with deterministic, bounded errors and derive rates for convergence and error accumulation in this setting.
In contrast, \citet{cohen2018acceleration} consider stochastic oracles with additive gradient noise and propose a noise-resistant acceleration scheme.
Very recently, \citet{chen2020understanding} analyze accelerated methods for strongly-convex functions under strong growth with additive noise. 

An alternative approach to provable stochastic acceleration using variance reduction for finite-sum functions was initiated by \citet{allen-zhu2017katyusha}.
Multiple algorithms of this type have been since proposed~\citep{allen-zhou2018katyushax, tang2018restkatyusha, kovalev2020loopless}.
Alternative approaches also leveraging finite-sum structure have been based on approximate proximal-point algorithms; such methods include Catalyst~\citep{lin2017catalyst} and accelerated APPA~\citep{frostig2015unregularizing}.

Several works have recently considered acceleration under interpolation.
The most similar to the investigation here is that of \citet{vaswani2019fast}, who analyze a slightly altered version of Nesterov's accelerated gradient descent under the strong growth condition. 
\citet{liu2020accelerating} propose a different modification, called MaSS, and analyze its convergence for convex quadratics as well as strongly-convex functions with additional assumptions. 
These assumptions imply strong growth, but yield hard-to-compare rates. 
\citet{jain2018accelerating} prove accelerated rates for squared-losses under interpolation using a tail-averaging scheme. 
Finally, \citet{assran2020convergence} study the stochastic approximation setting and prove that accelerated gradient descent may fail to accelerate even when interpolation is satisfied.\\

\noindent \textbf{Line Search}:
Line-search is a classic technique to set the step-size in deterministic optimization (see \citet{nocedal1999numerical}), but extensions to stochastic optimization are non-obvious. 
\citet{mahsereci2017pls} use a Gaussian process model to define probabilistic Wolfe conditions~\citep{wolfe1969convergence, wolfe1971convergence}; however, the convergence of \ac{SGD} with this procedure is not known.
\citet{fridovich2019choosing} propose line-search conditions based on golden-section search~\citep{avriel1968golden}, but again only provide empirical evidence for \ac{SGD}. 
\citet{paquette2020stochastic} and \citet{ogaltsov2019adaptive} both prove convergence of \ac{SGD} with the Armijo condition in the general stochastic setting with several caveats. 
\citet{paquette2020stochastic} require adaptive batch-sizes, while \citet{ogaltsov2019adaptive} use explicit knowledge of an upper-bound on variance in the stochastic gradients. 
Other authors consider similar approaches based on multiple function and/or gradient evaluations at each iteration~\citep{friedlander2012hybrid, byrd2012sample, de2016big, krejic2013line}.
Such approaches also use growing batch-sizes to ensure convergence.
The work in the thesis follows \citet{vaswani2019painless}, who investigated stochastic versions of the Armijo line-search under interpolation; we provide tighter analyses in a more general setting.\\ 


\noindent \textbf{Asymptotic Convergence}:
The original paper by \citet{robbins1951sgd} establishes asymptotic, almost-sure convergence for \ac{SGD} to the zero of a convex function if the stochastic gradients are bounded and a decreasing step-size is used.
Highly general analyses have since shown almost-sure convergence for non-convex functions when strong growth with additive noise is satisfied~\citep{bertsekas2000gradient, bottou1991approche}.
Alternative work directly derives these conditions from properties of strongly-convex and non-convex functions, also for the purpose of proving almost-sure convergence~\citep{nguyen2018sgd, lei2019stochastic}.
Recently, asymptotic convergence was shown with Adagrad-style step-sizes instead of a fixed, decreasing schedule~\citep{li2019convergence}.


\endinput


%    2. Main body
%!TEX root = ../main.tex

\chapter{Stochastic Gradient Descent}~\label{ch:sgd}


\section{Convergence for Strongly-Convex Functions}~\label{sec:sgd-sc}

We establish the convergence rate of SGD for strongly-convex under the strong growth condition. 
The following theorem allows for a larger step-size and establishes asymptotically faster convergence than the corresponding theorem in \citet{vaswani2019fast}.

\begin{restatable}{theorem}{sgcConvex}~\label{thm:sgc-convex}
    Let \( f \) be a \( \mu \)-strongly-convex, \( L \)-smooth function satisfying the strong growth condition with constant \( \rho \).
    Then stochastic gradient descent with step-size \( \eta \leq \frac{2}{\rho(\mu + L)} \) converges as 
    \[ f(\wk) - f(\wopt) \leq \rbr{\frac{L}{\mu}}\rbr{1 - \frac{2\eta \mu L}{\mu + L}}^k \rbr{f(\w_0) - f(\wopt)}. \] 
\end{restatable}

See \autoref{app:sgd-sc} for proof.
A key feature of this theorem is is that it requires only the full function \( f \) to be strongly-convex; the individual/stochastic functions \( f(\cdot, \z) \) may be merely convex, as is typically the case in the finite-sum setting.
Stochastic gradient descent is sub-optimal if the stochastic functions are also strongly-convex and the optimization procedure has direct access to \( f(\cdot, \z ) \) for each \( \z \).
In this case, gradient descent on the stochastic function with smallest condition number \( \kappa_{\text{min}} \) converges to the global minimize of the full function \( f \) as \( O\rbr{\exp\cbr{\frac{-4 k}{\kappa_{\text{min}} + 1}}} \).
Despite being a poor algorithmic choice, SGD will converge deterministically in this setting, as established in the following theorem.

\begin{restatable}{theorem}{sgcIndSC}~\label{thm:sgc-ind-sc}
    Let \( f \) be a \( \mu \)-strongly-convex, \( L \)-smooth function satisfying the strong growth condition with constant \( \rho \).
    Furthermore, assume that the stochastic functions \( f(\cdot, \z) \) are (i) \( \mu_\z \)-strongly-convex and (ii) \( L_\z \)-smooth.
    Let \( \mumax \) and \( \mumin \) be the largest and smallest strong-convexity constants for the stochastic functions, respectively.
    Similarly, let \( \Lmax \) and \( \Lmin \) be the largest and smallest Lipschitz constants of the stochastic gradients.
    Then stochastic gradient descent with step-size \( \eta \leq \frac{2}{\mumax + \Lmax} \) converges deterministically at the rate 
    \[ f(\wkk) - f(\wopt) \leq \frac{L}{\mu} \rbr{1 - \frac{2 \eta \mumin \Lmin}{\mumax + \Lmax}}^k \rbr{f(\w_0) - f(\wopt)}. \] 
\end{restatable}


\section{Convergence for Convex Functions}~\label{sec:sgd-convex}
\todo{Clarify the improvements in this proof}

We establish the convergence rate of SGD for convex functions under the weak growth condition. These results improve on those given by \citet{vaswani2019fast} by constant factors. 
Moreover, we provide cleaner and simpler proofs.


\begin{restatable}{theorem}{wgcConvex}~\label{thm:wgc-convex}
    Let \( f \) be a convex, \( L \)-smooth function satisfying the weak growth condition with constant \( \rho  \).
    Then stochastic gradient descent with step-size \( \eta < \frac{1}{\rho L} \) converges as
    \[ \f(\bar \w_K) - f(\wopt) \leq \frac{1}{2 \eta \rbr{1 - \eta \rho L} \, K} \norm{\w_0 - \wopt}^2, \]
    where \( \bar \w_K = \frac{1}{K} \sum_{k=0}^K \wk \). 
\end{restatable}

A slightly larger step-size can be obtained if we further assume that the stochastic functions are themselves Lipschitz-smooth.

\begin{restatable}{theorem}{wgcConvexIndSmooth}~\label{thm:wgc-convex-ind-smooth}
    Let \( f \) be a convex, \( L \)-smooth function satisfying the weak growth condition with constant \( \rho  \).
    Moreover, suppose that the stochastic functions \( f(\cdot, \z) \) are \( L_\z \)-smooth for each \( \z \), with \( \Lmax = \max_\z L_\z \).
    Then stochastic gradient descent with step-size \( \eta < \frac{1}{\rho L} + \frac{1}{\Lmax} \) converges as
    \[ \E\sbr{f(\bar w_K)} - f(\wopt) \leq \frac{1}{2\eta \, C \, K} \norm{\w_0 - \wopt}^2,   \]
    where \( \bar \w_K = \frac{1}{K} \sum_{k=0}^K \wk \) and \( C = 1 + \min \cbr{ 0, \rho L \rbr{\frac{1}{\Lmax} - \eta }} \). 
\end{restatable}

See \autoref{app:sgd-convex} for proof.

\section{Almost Sure Convergence}

Our overall goal is to characterize the asymptotic behavior of SGD.
In particular, we want to show that the random variable \( \lim_{\iter \rightarrow \infty} \norm{\grad(\wk)}^2 \) exists and is almost-surely zero.
We will need some tools from measure-theoretic probability to accomplish this.
To start, we note that each iterate \( \w_K \) can be written as deterministic function of the stochastic gradients \( \cbr{\grad(\wk, \zk)}_{\iter=0}^{K} \) by unrolling the SGD update.
Formally, we assume a probability space \( \rbr{\Omega,  \calF, P} \) in the background and say the sequence \( \seq{\wk} \) is \emph{adapted} to the filtration generated by the stochastic gradients,
\begin{align*}
    \calF_K = \sigma \rbr{\bigcup_{\iter = 0}^{K-1} \sigma \grad(\wk, \zk)}.
\end{align*}
The sequences of function and gradient values are functions of \( \seq{\wk} \) and so are also adapted to \( \seq{\calF_\iter} \).
See \citet{ccinlar2011probability} additional details on filtrations.

Our main tool to show convergence of sequences of random variables will be supermartingale theory~\citep{}.
Supermartingales are one of two classic tools used analyze the convergence of SGD, the other being Lyapunov functions~\citep{bertsekas2000gradient}.
In particular, recent authors have made use of convergence of discrete-time, positive supermartingales~\citep{bertsekas2011incremental, nguyen2018sgd}.
This theorem is due to~\citet{neveu1975discrete} and will be the cornerstone of our analyses; we reproduce it here for convenience.

\begin{theorem}[Convergence of Positive Supermartingales]\label{thm:positive_supermartingales}
    Let \( \seq{Y_\iter} \), \( \seq{X_\iter} \), and \( \seq{A_\iter} \) be discrete, non-negative random processes indexed by \( \iter \in \bbN \) and adapted to the filtration \( \seq{\calF_\iter} \).
    Suppose that
    \begin{align*}
        \forall \iter \in \bbN, \: \E \sbr{Y_{\iter+1} \mid \calF_\iter} \leq Y_\iter - X_\iter + A_\iter,
        && \text{ and } &&
        \sum_{k=0}^{\infty} A_\iter < \infty,
    \end{align*}
    almost surely.
    Then the sequence \( \seq{Y_\iter} \) converges almost surely to a non-negative random variable \( Y_\infty \) and \( \sum_{k=0}^{\infty} X_\iter < \infty \) almost surely.
\end{theorem}
With this necessary martingale theorem in hand, we are now ready to study the almost-sure convergence of stochastic gradient descent.

\begin{restatable}{theorem}{wgcAlmostSure}~\label{theorem:wgc-almost-sure}
    Let \( \f \) be an \( L \)-smooth, convex function that is bounded below.
    Assume that \( f \) has at least one finite minimizer \( \wopt \).
    Moreover, suppose that the stochastic functions \( f(\cdot, \z) \) are \( L_\z \)-smooth for each \( \z \), with \( \Lmax = \max_\z L_\z \).
    If \( f \) satisfies the weak growth condition with parameter \( \rho \), then stochastic gradient descent with step-size \( \eta < \frac{1}{\rho L} + \frac{1}{\Lmax} \) converges to \( \fopt \) almost surely.
\end{restatable}

\todo{note that the conditions can be relaxed not avoid individual smoothness at the cost of a larger step-size.}

\begin{restatable}{theorem}{sgcAlmostSure}~\label{theorem:sgc-almost-sure}
    Let \( \f \) be an \( L \)-smooth function that is bounded below and satisfies the strong growth condition with parameter \(\rho \).
    Then stochastic gradient descent with fixed step-size \(\eta < \frac{2}{\rho L} \) converges to a stationary point almost surely.
\end{restatable}

\endinput


%    3. Notes
%    4. Footnotes

%    5. Bibliography
\begin{singlespace}
\raggedright
\bibliographystyle{abbrvnat}
\bibliography{refs}
\end{singlespace}

\appendix
%    6. Appendices (including copies of all required UBC Research
%       Ethics Board's Certificates of Approval)
%\include{reb-coa}	% pdfpages is useful here
%!TEX root = ../main.tex

\chapter{Supporting Materials}


%! TEX root = ../main.tex

\section{SGD: Proofs}~\label{app:sgd}

\subsection{Convergence for Strongly Convex Functions}~\label{app:sgd-sc}

\begin{restatable}{lemma}{sgcDecreaseCondition}~\label{lemma:sgc-decrease-condition}
    Let \( f \) be an \( L \)-smooth function satisfying the strong growth condition with parameter \( \rho \).
    Then stochastic gradient descent with fixed step-size \( \eta \) satisfies the following expected decrease condition:
    \[ \Ek \sbr{f(\wkk)} \leq f(\wk) - \eta \rbr{1 - \frac{\rho L \eta}{2}}\norm{\grad(\wk)}^2. \]
\end{restatable}

\begin{proof}
    Starting from \( L \)-smoothness of \( f \),
    \begin{align*}
        f(\wkk) &\leq f(\wk) + \abr{\grad(\wk), \wkk - \wk} + \frac{L}{2}\norm{\wkk - \wk}^2\\
        &= f(\wk) - \eta \abr{\grad(\wk), \grad(\wk, \zk)} + \frac{L \eta^2}{2}\norm{\grad(\wk, \zk)}^2\\
        \intertext{Taking expectations with respect to \( \zk \):}
        \implies \Ek[f(\wkk)] &\leq f(\wk) - \eta \norm{\grad(\wk)}^2 + \frac{L \eta^2}{2}\Ek \sbr{\norm{\grad(\wk, \zk)}^2}.
        \intertext{Using the strong growth condition,}
        \implies f(\wkk) - f(\wk) &\leq f(\wk) - \eta \norm{\grad(\wk)}^2 + \frac{\rho L \eta^2}{2} \norm{\grad(\wk)}^2\\
        &\leq f(\wk) - \eta \rbr{1 - \frac{\rho L \eta}{2}}\norm{\grad(\wk)}^2.
    \end{align*}
\end{proof}


\sgcConvex*
\begin{proof}
    \begin{align*}
        \norm{\wkk - \wopt}^2 &= \norm{\wk - \eta \grad(\wk, \zk) + - \wopt}^2\\
                             &= \eta^2 \norm{\grad(\wk, \zk)}^2 - 2 \eta \abr{\grad(\wk, \zk), \wk - \wopt} + \norm{\wk - \wopt}^2.
                             \intertext{Taking expectations with respect to \( \zk \), }
       \implies \Ek \sbr{\norm{\wkk - \wopt}^2} &= \eta^2 \Ek \sbr{\norm{\grad(\wk, \zk)}^2} - 2 \eta \Ek \sbr{\abr{\grad(\wk, \zk), \wk - \wopt}} + \norm{\wk - \wopt}^2\\
                                      &= \eta^2 \Ek \sbr{\norm{\grad(\wk, \zk)}^2} - 2 \eta \abr{\grad(\wk), \wk - \wopt} + \norm{\wk - \wopt}^2.
                                      \intertext{Now we use the strong growth condition to control \( \Ek \sbr{\norm{\grad(\wk, \zk)}^2} \), which yields}
       \Ek\sbr{\norm{\wkk - \wk}^2} &\leq \eta^2 \rho \norm{\grad(\wk)}^2 - 2 \eta \abr{\grad(\wk), \wk - \wopt} + \norm{\wk - \wopt}^2.
                                      \intertext{Coercivity of the gradient (\autoref{lemma:coercivity}) implies}
       \Ek\sbr{\norm{\wkk - \wk}^2} &\leq \eta^2 \rho \norm{\grad(\wk)}^2 - 2 \eta \rbr{\frac{\mu L }{\mu + L} \norm{\wk - \wopt}^2 + \frac{1}{\mu + L}\norm{\grad(\wk)}^2} + \norm{\wk - \wopt}^2\\
                                   &= \eta \rbr{\eta \rho  - \frac{2}{\mu + L}}\norm{\grad(\wk)}^2 + \rbr{1 - \frac{2 \eta \mu L}{\mu + L}}\norm{\wk - \wopt}^2.
                                   \intertext{If \( \eta \leq \frac{2}{\rho \rbr{\mu + L}} \) then \( \eta \rho - \frac{2}{\mu + L} \leq 0 \) and we obtain}
       \Ek \sbr{\norm{\wkk - \wopt}^2} &\leq \rbr{1 - \frac{2 \eta \mu L}{\mu + L}}\norm{\wk - \wopt}^2.
       \intertext{Taking expectations and recursing on this inequality,}
       \implies \E \sbr{\norm{\wkk - \wopt}^2} &\leq \rbr{1 - \frac{2 \eta \mu L}{\mu + L}}^k \norm{\w_0 - \wopt}^2.
       \intertext{\autoref{lemma:iterate-bounds} now completes the proof:}
       \implies \E \sbr{f(\wkk)} - f(\wopt) &\leq \frac{L}{\mu} \rbr{1 - \frac{2 \eta \mu L}{\mu + L}}^k \rbr{f(\w_0) - f(\wopt)}.
    \end{align*}
\end{proof}

\sgcIndSC*
\begin{proof}
    \begin{align*}
        \norm{\wkk - \wopt}^2 &= \norm{(\wkk - \wk) + (\wk - \wopt)}^2\\
                             &= \norm{\wkk - \wk}^2 + 2 \abr{\wkk - \wk, \wk - \wopt} + \norm{\wk - \wopt}^2\\
                             &= \eta^2 \norm{\grad(\wk, \zk)}^2 - 2 \eta \abr{\grad(\wk, \zk), \wk - \wopt} + \norm{\wk - \wopt}^2.
                            \intertext{Coercivity of the stochastic gradient (\autoref{lemma:coercivity}) implies}
        \norm{\wkk - \wk}^2 &\leq \eta^2 \norm{\grad(\wk, \zk)}^2 - 2 \eta \rbr{\frac{\mu_\z L_\z }{\mu_\z + L_\z} \norm{\wk - \wopt}^2 + \frac{1}{\mu_\z + L_\z}\norm{\grad(\wk, \zk)}^2} \\ & \hspace{2cm} + \norm{\wk - \wopt}^2\\
                            &\leq \eta^2 \norm{\grad(\wk, \zk)}^2 - 2 \eta \rbr{\frac{\mumin \Lmin }{\mumax + \Lmax} \norm{\wk - \wopt}^2 + \frac{1}{\mumax + \Lmax}\norm{\grad(\wk, \zk)}^2} \\ & \hspace{2cm} + \norm{\wk - \wopt}^2\\
                                   &= \eta \rbr{\eta - \frac{2}{\mumax + \Lmax}}\norm{\grad(\wk, \zk)}^2 + \rbr{1 - \frac{2 \eta \mumin \Lmin}{\mumin + \Lmax}}\norm{\wk - \wopt}^2.
                                   \intertext{If \( \eta \leq \frac{2}{\rbr{\mumax + \Lmax}} \) then \( \eta - \frac{2}{\mumax + \Lmax} \leq 0 \) and we obtain}
       \norm{\wkk - \wopt}^2 &\leq \rbr{1 - \frac{2 \eta \mumin \Lmin}{\mumax + \Lmax}}\norm{\wk - \wopt}^2.
       \intertext{Recursing on this inequality,}
       \implies \norm{\wkk - \wopt}^2 &\leq \rbr{1 - \frac{2 \eta \mumin \Lmin}{\mumax + \Lmax}}^k \norm{\w_0 - \wopt}^2.
       \intertext{Application of \autoref{lemma:iterate-bounds} completes the proof,}
       \implies f(\wkk) - f(\wopt) &\leq \frac{L}{\mu} \rbr{1 - \frac{2 \eta \mumin \Lmin}{\mumax + \Lmax}}^k \rbr{f(\w_0) - f(\wopt)}.
    \end{align*}
\end{proof}

\subsection{Convergence for Convex Functions}~\label{app:sgd-convex}

\begin{restatable}{lemma}{convexIntermediate}~\label{lemma:convex-intermediate}
    Let \( f \) be a convex, \( L \)-smooth function.
    Moreover, assume that the stochastic functions \( \f(\cdot, \z) \) are \( L_\z \)-smooth.
    Then stochastic gradient descent with step-size \( \eta \) satisfies the following inequality: 
    \[ f(\wk) - f(\wopt) \leq \frac{1}{2\eta \, \delta} \rbr{\norm{\wk - \wopt}^2 - \Ek \sbr{\norm{\wkk - \wopt}^2}}, \]
    where \(  \delta = \min \cbr{ 1, 1 + \rho L \rbr{\frac{1}{\Lmax} - \eta }} \). 
    Furthermore, if \( \eta \leq \frac{1}{\Lmax} \), then 
    \[ f(\wk) - f(\wopt) \leq \frac{1}{2 \eta}\rbr{\norm{\wk - \wopt}^2 - \Ek\sbr{\norm{\wkk - \wopt}^2}}. \]
\end{restatable}

\begin{proof}
    \begin{align*}
       \norm{\wkk - \wopt}^2 &= \norm{\wk - \etak \grad(\wk, \zk) + - \wopt}^2\\
                             &= \eta^2 \norm{\grad(\wk, \zk)}^2 - 2 \eta \abr{\grad(\wk, \zk), \wk - \wopt} + \norm{\wk - \wopt}^2. 
                             \intertext{The weak growth condition implies \( \grad(\wopt, \z) = 0 \) for all \( \z \). We may thus use \autoref{lemma:pre-coercivity} at \( \wk \) and \( \wopt \) to obtain}
                       \norm{\wkk - \wopt}^2 &\leq \eta^2 \norm{\grad(\wk, \zk)}^2 - 2 \eta \rbr{f(\wk, \zk) - f(\wopt, \zk) + \frac{1}{2L_\z} \norm{\grad(\wk, \zk)}^2}  + \norm{\wk - \wopt}^2\\ 
                                             &\leq \eta^2 \norm{\grad(\wk, \zk)}^2 - 2 \eta \rbr{f(\wk, \zk) - f(\wopt, \zk) + \frac{1}{2\Lmax} \norm{\grad(\wk, \zk)}^2}  + \norm{\wk - \wopt}^2\\
                         &\leq \rbr{\eta^2 - \frac{\eta}{\Lmax}} \norm{\grad(\wk, \zk)}^2 - 2 \eta \rbr{f(\wk, \zk) - f(\wopt, \zk)} + \norm{\wk - \wopt}^2.
     \intertext{Taking expectations with respect to \( \zk \):}
                      \Ek \sbr{\norm{\wkk - \wopt}^2} &\leq \rbr{\eta^2 - \frac{\eta}{\Lmax}} \Ek \sbr{\norm{\grad(\wk, \zk)}^2} - 2 \eta \Ek\sbr{f(\wk, \zk) - f(\wopt, \zk)} + \norm{\wk - \wopt}^2,\\
                         &\leq \rbr{\eta^2 - \frac{\eta}{\Lmax}} \Ek \sbr{\norm{\grad(\wk, \zk)}^2} - 2 \eta \rbr{f(\wk) - f(\wopt)} + \norm{\wk - \wopt}^2.
                     \intertext{\textbf{Case 1}: If \( \eta \leq \frac{1}{\Lmax} \) then \( \rbr{\eta^2 - \frac{\eta}{\Lmax}}\rbr{f(\wk, \zk) - f(\wopt, \zk)} \leq 0 \) by interpolation and we obtain }
                     \Ek \sbr{\norm{\wkk - \wopt}^2} &\leq - 2 \eta\rbr{f(\wk) - f(\wopt)} + \norm{\wk - \wopt}^2\\
                     \implies f(\wk) - f(\wopt) &\leq \frac{1}{2\eta} \sbr{\norm{\wk - \wopt}^2 - \Ek \norm{\wkk - \wopt}^2}. 
                     \intertext{\textbf{Case 2}: If \( \eta > \frac{1}{\Lmax} \) then \( \eta^2 - \frac{\eta}{\Lmax} > 0 \) and the weak growth condition implies}
                      \Ek \sbr{\norm{\wkk - \wopt}^2} &\leq 2 \eta \rho L \rbr{\eta - \frac{1}{\Lmax}}\rbr{\f(\wk) - f(\wopt)} - 2 \eta\rbr{f(\wk) - f(\wopt)} + \norm{\wk - \wopt}^2\\ 
                                           &= -2 \eta \rbr{1 + \rho L \rbr{\frac{1}{\Lmax} - \eta }}\rbr{f(\wk) - f(\wopt)} + \norm{\wk - \wopt}^2.
                     \intertext{If \( \eta < \inv{\rho L} + \inv{\Lmax} \), then \( 1 + \rho L \rbr{\frac{1}{\Lmax} - \eta } > 0 \), }
                      \implies f(\wk) - f(\wopt) &\leq \frac{1}{2 \eta \rbr{1 + \rho L \rbr{\frac{1}{\Lmax} - \eta}}} \rbr{\norm{\wk - \wopt}^2 - \Ek\sbr{\norm{\wkk - \wopt}^2}}. 
          \intertext{Let us combine the cases by taking the worst-case bound. 
                     Let \( \delta = \min \cbr{ 1,1 + \rho L \rbr{\frac{1}{\Lmax} - \eta }} \) to obtain: }
    f(\wk) - f(\wopt) &\leq \frac{1}{2\eta \, \delta} \rbr{\norm{\wk - \wopt}^2 - \Ek \sbr{\norm{\wkk - \wopt}^2}}.
\end{align*}
Note that this bound is tight with Case 1 since \( 1 + \rho L \rbr{\inv{\Lmax} - \eta} \geq 1\) when \( \eta \leq \inv{\Lmax} \). 
\end{proof}



\wgcConvex*
\begin{proof}
    \begin{align*}
        \norm{\wkk - \wopt}^2 &= \norm{\wk - \eta \grad(\wk, \zk) - \wopt}^2\\
                             &= \eta^2 \norm{\grad(\wk, \zk)}^2 - 2 \eta \abr{\grad(\wk, \zk), \wk - \wopt} + \norm{\wk - \wopt}^2.
                             \intertext{Taking expectations with respect to \( \zk \), }
       \Ek \sbr{\norm{\wkk - \wopt}^2} &= \eta^2 \Ek \sbr{\norm{\grad(\wk, \zk)}^2} - 2 \eta \Ek \sbr{\abr{\grad(\wk, \zk), \wk - \wopt}} + \norm{\wk - \wopt}^2\\
                                      &= \eta^2 \Ek \sbr{\norm{\grad(\wk, \zk)}^2} - 2 \eta \abr{\grad(\wk), \wk - \wopt} + \norm{\wk - \wopt}^2.
                             \intertext{By convexity of \( f \) and the weak growth condition,}
       \Ek \sbr{\norm{\wkk - \wopt}^2} &\leq \eta^2 \norm{\grad(\wk, \zk)}^2 - 2 \eta \rbr{f(\wk) - f(\wopt)} + \norm{\wk - \wopt}^2\\
                             &\leq 2 \eta^2 \rho L \rbr{f(\wk) - f(\wopt)} - 2 \eta \rbr{f(\wk) - f(\wopt)} + \norm{\wk - \wopt}^2\\
                             &= - 2 \eta \rbr{1 - \eta \rho L}\rbr{f(\wk) - f(\wopt)} + \norm{\wk - \wopt}^2.\addnumber~\label{eq:cwg-alternate-progress}
   \end{align*}
   Rearranging the expression to put the optimality gap on the left-hand side, 
   \begin{align*}
       2 \eta \rbr{1 - \eta \rho L} \rbr{f(\wk) - f(\wopt)} &\leq \norm{\wk - \wopt}^2 - \Ek \sbr{\norm{\wkk - \wopt}^2}.
       \intertext{If \( \eta < \frac{1}{\rho L} \) then \( 1 - \eta \rho L > 0 \), }
       \implies f(\wk) - f(\wopt) &\leq \frac{1}{2 \eta \rbr{1 - \eta \rho L}} \rbr{\norm{\wk - \wopt}^2 - \Ek\sbr{\norm{\wkk - \wopt}^2}}.
       \intertext{Taking expectations and summing from \( k = 0 \) to \( K - 1 \) now gives}
   \frac{1}{K} \sum_{k=0}^{K-1} \E \sbr{f(\wk)} - f(\wopt) &\leq \frac{1}{2 \eta \rbr{1 - \eta \rho L} \, K} \sum_{k=0}^{K-1} \rbr{\E \sbr{\norm{\wk - \wopt}^2} - \E \sbr{\norm{\wkk - \wopt}^2}}\\
                                                           &\leq \frac{1}{2 \eta \rbr{1 - \eta \rho L} \, K} \rbr{\norm{\w_0 - \wopt}^2 - \norm{\w_K - \wopt}^2}\\
                                                           &\leq \frac{1}{2 \eta \rbr{1 - \eta \rho L} \, K} \norm{\w_0 - \wopt}^2.
                                                           \intertext{Noting \( \frac{1}{K} \sum_{k=0}^{K-1} f(\wk) \geq f(\bar \w_K) \) by convexity leads to the final result,}
   \E\sbr{f(\bar \w_K)} - f(\wopt) &\leq \frac{1}{2 \eta \rbr{1 - \eta \rho L} \, K} \norm{\w_0 - \wopt}^2.
   \end{align*}
\end{proof}


\wgcConvexIndSmooth*

\begin{proof}
   Starting from \autoref{lemma:convex-intermediate},
   \begin{align*}
    f(\wk) - f(\wopt) &\leq \frac{1}{2\eta \, C} \rbr{\norm{\wk - \wopt}^2 - \Ek \sbr{\norm{\wkk - \wopt}^2}}\\
    \intertext{Taking expectations and summing from \( k = 0 \) to \( K - 1 \) now gives}
    \implies \frac{1}{K} \sum_{k=0}^{K -1} \E \sbr{f(\wk) - f(\wopt)} &\leq \frac{1}{2\eta \, \delta \, K}\sum_{k=0}^{K-1}\rbr{\E\sbr{\norm{\wk - \wopt}^2} - \E\sbr{\norm{\wkk - \wopt}^2}}\\
                                                         &= \frac{1}{2\eta \, \delta \, K}\rbr{\norm{\w_0 - \wopt}^2 - \E\sbr{\norm{\w_{K} - \wopt}^2}}\\
                                                         &\leq \frac{1}{2\eta \, \delta \, K} \norm{\w_0 - \wopt}^2\\
\intertext{Noting \( \frac{1}{K}\sum_{k=0}^{K-1} f(\wk) \geq f(\bar \w_K) \) by convexity leads to the final result,}
   \implies \E\sbr{f(\bar w_K)} - f(\wopt) &\leq \frac{1}{2\eta \, \delta \, K}\norm{\w_0 - \wopt}^2.
\end{align*}
\end{proof}

\todo{Can we prove a final iterate version of this using the strategy in Bubeck's ConvexOpt? If not, can we prove that SGD is not guaranteed to make progress at every iteration even when WGC holds?}

\subsection{Almost Sure Convergence}~\label{app:almost-sure-convergence}

\wgcAlmostSure*
\begin{proof}
    Lemma~\ref{lemma:convex-intermediate} gives the decrease condition
    \begin{align*}
        \E\sbr{\norm{\wkk - \wopt}^2 \mid \calF_\iter} \leq  \norm{\wk - \wopt}^2 - 2 \eta \delta \rbr{f(\wk) - f(\wopt)},
    \end{align*}
    where \( \delta \geq 1 \) since \( \eta < \frac{1}{\rho L} + \frac{1}{\Lmax} \)
    The conditions of Theorem~\ref{thm:positive_supermartingales} are satisfied with \( A_\iter = 0 \) for all \( \iter \) and implies sequence \( \seq{\norm{\wk - \wopt}^2} \) converges to a non-negative random variable \(\lim_{\iter \rightarrow \infty} \norm{\wk - \wopt}^2 \) almost surely.
    The theorem also guarantees
    \begin{align}
        \sum_{k=0}^{\infty} 2\eta \delta \rbr{f(\wk) - f(\wopt)} &< \infty \nonumber \\
        \implies \sum_{k=0}^{\infty} \rbr{f(\wk) - f(\wopt)} &< \infty~\label{eq:func_convergence}\\
        \implies \lim_{k \rightarrow \infty} f(\wk) = f(\wopt). \nonumber
    \end{align}
    almost surely.
    That is, stochastic gradient descent converges to the optimal function value. 
    Let us extend this convergence to the sequence of iterates \( \seq{\wk} \).
    The weak growth condition implies 
    \begin{align*}
        \E\sbr{\sum_{k=0}^{K} \norm{\wkk - \wk}^2} &= \E\sbr{\sum_{k=0}^{K} \eta^2 \norm{\grad(wk, \zk)}^2}\\
                                                   &\leq \sum_{k=0}^{K} \rho \eta^2 \E\sbr{f(\wk) - f(\wopt)}\\
        \implies \lim_{K\rightarrow \infty} \E \sbr{\sum_{k=0}^{K} \norm{\wkk - \wk}^2} &\leq \lim_{K\rightarrow\infty} \E \sbr{\sum_{k=0}^{K} \rho \eta^2 \rbr{f(\wk) - f(\wopt)}}\\
        \intertext{The partial sums are increasing \( K \), so we may apply the monotone convergence theorem (MCT) to obtain} 
        \E \sbr{\sum_{k=0}^{\infty} \norm{\wkk - \wk}^2} &\leq  \E \sbr{\sum_{k=0}^{\infty} \rho \eta^2 \rbr{f(\wk) - f(\wopt)}}\\
                                                                  &< \infty. \tag{by \autoref{eq:func_convergence}}
    \end{align*}
    The series \(  \sum_{k=0}^{\infty} \norm{\wkk - \wk}^2 \) is a non-negative random variable with finite expectation and so converges almost surely: 
    \[ \sum_{k=0}^{\infty} \norm{\wkk - \wk}^2 < \infty. \]
    Thus, \( \lim_{k \rightarrow \infty} \norm{\wkk - \wk}^2 \equas 0 \) and the Cauchy criterion now guarantees \( \lim_{k \rightarrow \infty} \wk = \w_\infty \) for some random variable \( \w_\infty \).
    Continuity of \( f \) and the continuous mapping theorem imply 
    \[ f(\w_\infty) \equas \lim_{k\rightarrow \infty} f(\wk) \equas f(\wopt). \]
    We conclude \( \w_\infty \in \argmin_\w f(\w) \) almost surely.
\end{proof}

\sgcAlmostSure*
\begin{proof}
    Lemma~\ref{lemma:sgc-decrease-condition} gives the decrease condition
    \begin{align*}
        \E \sbr{f(\wkk) - f(\wopt) \mid \calF_\iter} &\leq \rbr{f(\wk) - f(\wopt)} - \eta \rbr{1 - \frac{\rho L \eta}{2}}\norm{\grad(\wk)}^2.
    \end{align*}
    Since \( \eta < \frac{2}{\rho L} \),
    \begin{align*}
        \eta \rbr{1 - \frac{\eta \rho L}{2}}\norm{\grad(\wk)}^2 > 0,
    \end{align*}
    and the conditions of Theorem~\ref{thm:positive_supermartingales} are satisfied with \( A_\iter = 0 \) for all \( \iter \).
    The sequence \( \seq{f(\wk) - f(\wopt)} \) converges almost surely to a non-negative random variable \(\lim_{\iter \rightarrow \infty} f(\wk) - f(\wopt) \) almost surely.
    Of more interest is that
    \begin{align}
        \sum_{k=0}^{\infty} \eta \rbr{1 - \frac{\eta \rho L}{2}} \norm{\grad(\wk)}^2 &< \infty \nonumber\\
        \implies \sum_{k=0}^{\infty} \norm{\grad(\wk)}^2 &< \infty, \label{eq:stationary-convergence}
    \end{align}
    almost surely.
    Accordingly, the sequence of gradient norms satisfies
    \[ \lim_{\iter \rightarrow \infty} \norm{\grad(\wk)}^2 \equas 0, \]
    and we conclude that the sequence of gradients converges almost surely to a stationary point.
   
    Let us extend this convergence to the sequence of iterates \( \seq{\wk} \).
    The strong growth condition implies 
    \begin{align*}
        \E\sbr{\sum_{k=0}^{K} \norm{\wkk - \wk}^2} &= \E\sbr{\sum_{k=0}^{K} \eta^2 \norm{\grad(wk, \zk)}^2}\\
                                                   &\leq \sum_{k=0}^{K} \rho \eta^2 \E\sbr{\norm{\grad(wk, \zk)}^2}\\
        \implies \lim_{K\rightarrow \infty} \E \sbr{\sum_{k=0}^{K} \norm{\wkk - \wk}^2} &\leq \lim_{K\rightarrow\infty} \E \sbr{\sum_{k=0}^{K} \rho \eta^2 \norm{\grad(wk, \zk)}^2}\\
        \intertext{The partial sums are increasing \( K \), so we may apply the monotone convergence theorem (MCT) to obtain} 
        \E \sbr{\sum_{k=0}^{\infty} \norm{\wkk - \wk}^2} &\leq  \E \sbr{\sum_{k=0}^{\infty} \rho \eta^2 \norm{\grad(wk, \zk)}^2}\\
                                                                  &< \infty. \tag{by \autoref{eq:stationary-convergence}}
    \end{align*}
    The series \(  \sum_{k=0}^{\infty} \norm{\wkk - \wk}^2 \) is a non-negative random variable with finite expectation and so converges almost surely: 
    \[ \sum_{k=0}^{\infty} \norm{\wkk - \wk}^2 < \infty. \]
    Thus, \( \lim_{k \rightarrow \infty} \norm{\wkk - \wk}^2 \equas 0 \) and the Cauchy criterion now guarantees \( \lim_{k \rightarrow \infty} \wk = \w_\infty \) for some random variable \( \w_\infty \).
    The mapping \( \w \mapsto \norm{\grad(\w)}^2 \) is a composition of continuous functions and thus is continuous. 
    The continuous mapping theorem implies  
    \[ \norm{\grad(\w_\infty)}^2  \equas \lim_{k\rightarrow \infty} \norm{\grad(\wk)}^2 \equas 0, \]
    which completes the proof.
\end{proof}











%% Old Proof with WGC

\iffalse
\begin{proof}
   \begin{align*}
       \norm{\wkk - \wopt}^2 &= \norm{(\wkk - \wk) + (\wk - \wopt)}^2\\
                             &= \norm{\wkk - \wk}^2 + 2 \abr{\wkk - \wk, \wk - \wopt} + \norm{\wk - \wopt}^2\\
                             &= \eta^2 \norm{\grad(\wk, \zk)}^2 - 2 \eta \abr{\grad(\wk, \zk), \wk - \wopt} + \norm{\wk - \wopt}^2\\ 
                             \intertext{The weak growth condition implies \( \grad(\wopt, \z) = 0 \) for all \( \z \). We may thus use co-coercivity of the gradient (\autoref{lemma:co-coercivity}) at \( \wk \) and \( \wopt \) to obtain}
                             &\leq \eta^2 \norm{\grad(\wk, \zk)}^2 + 2 \eta \rbr{f(\wopt, \zk) - f(\wk, \zk) - \frac{1}{2L} \norm{\grad(\wk, \zk)}^2}  + \norm{\wk - \wopt}^2\\ 
                             &\leq \rbr{\eta^2 - \frac{\eta}{L}} \norm{\grad(\wk, \zk)}^2 + 2 \eta \rbr{f(\wopt, \zk) - f(\wk, \zk)} + \norm{\wk - \wopt}^2\\
                             \intertext{Taking expectations with respect to \( \zk \):}
                             &\leq \rbr{\eta^2 - \frac{\eta}{L}} \E \sbr{\norm{\grad(\wk, \zk)}^2} + 2 \eta \E\sbr{f(\wopt, \zk) - f(\wk, \zk)} + \norm{\wk - \wopt}^2\\
                             &\leq \rbr{\eta^2 - \frac{\eta}{L}} \E \sbr{\norm{\grad(\wk, \zk)}^2} + 2 \eta \rbr{f(\wopt) - f(\wk)} + \norm{\wk - \wopt}^2\\
                             \intertext{If \( \eta \leq \frac{1}{L} \) then \( \eta^2 - \frac{\eta}{L} \geq 0 \) and we apply the weak growth condition as follows: }
                             &\leq 2 \rho \rbr{\eta^2- \frac{\eta}{L}}\rbr{\f(\wk) - f(\wopt)} + 2 \eta\rbr{f(\wopt) - f(\wk)} + \norm{\wk - \wopt}^2\\ 
                             &= 2\eta \rho \rbr{\eta - \frac{1}{L} - \frac{1}{\rho}}\rbr{f(\wk) - f(\wopt)} + \norm{\wk - \wopt}^2
\end{align*}
Rearranging the expression to put the optimality gap on the left-hand side,
\begin{align*}
    2 \eta \rho \rbr{\frac{1}{L} + \frac{1}{\rho} - \eta}\rbr{f(\wk) - f(\wopt)} &\leq \norm{\wk - \wopt}^2 - \E\sbr{\norm{\wkk - \wopt}^2}\\ 
\intertext{Noting that \( \frac{1}{L} + \frac{1}{\rho} - \eta \geq 0 \) since \( \eta \leq \frac{1}{L} \), }
\implies f(\wk) - f(\wopt) &\leq \rbr{\frac{L + \rho - \eta \rho L}{2\eta\rho^2L}} \rbr{\norm{\wk - \wopt}^2 - \norm{\wkk - \wopt}^2}\\
\intertext{Taking expectations and summing over iterations,}
\implies \sum_{k=0}^{K -1} \E \sbr{f(\wk) - f(\wopt)} &\leq \sum_{k=0}^{K-1}\rbr{\frac{L + \rho - \eta \rho L}{2\eta\rho^2L}} \rbr{\E\sbr{\norm{\wk - \wopt}^2} - \E\sbr{\norm{\wkk - \wopt}^2}}\\
                                                      &\leq \rbr{\frac{L + \rho - \eta \rho L}{2\eta\rho^2L}}\rbr{\norm{\w_0 - \wopt}^2 - \E\sbr{\norm{\w_{K} - \wopt}^2}}\\
                                                      &\leq \rbr{\frac{L + \rho - \eta \rho L}{2\eta\rho^2L}}\norm{\w_0 - \wopt}^2\\
\implies \frac{1}{K} \sum_{k=0}^{K-1} \E\sbr{f(\wk)} - f(\wopt) &\leq \rbr{\frac{L + \rho - \eta \rho L}{2K\eta\rho^2L}}\norm{\w_0 - \wopt}^2.
\intertext{Noting \( \frac{1}{K}\sum_{k=0}^{K-1} f(\wk) \geq f(\bar \wk) \) by convexity gives the final result,}
\implies \E\sbr{f(\bar w_K)} - f(\wopt) &\leq \rbr{\frac{L + \rho -\eta \rho L}{2K\eta \rho^2 L}}\norm{\w_0 - \wopt}^2.
\end{align*}
\end{proof}
\fi




\backmatter
%    7. Index
% See the makeindex package: the following page provides a quick overview
% <http://www.image.ufl.edu/help/latex/latex_indexes.shtml>


\end{document}
